\section{Work organization} % (fold)
\label{sec:work_organization}


Here is a sample organization for working with \tkzNamePack{tkz-euclide} and \tkzEngine{lua\LATEX}.

\begin{itemize}
  \item \textbf{Compilation:} Add the line:
  \begin{verbatim}
  % !TEX TS-program = lualatex
  \end{verbatim}
  to ensure that your document compiles with \tkzEngine{lua\LATEX}.

  \item \textbf{Document Class:} The \texttt{standalone} class is recommended when the goal is simply to create a figure, as it avoids unnecessary overhead.

  \item \textbf{Package Loading:} You can load \tkzNamePack{tkz-euclide} in two ways:
  \begin{itemize}
    \item \tkzcname{usepackage\{tkz-euclide\} }gives you full access to the entire package.
    \item The recommended method is to use the \tkzPopt{tkz-euclide}{mini} option, which loads only the necessary modules for drawing. You still retain the ability to draw with \TIKZ{} if desired.
  \end{itemize}

  \item \textbf{Conditionals:} The package \tkzNamePack{ifthen} is useful when you need to evaluate Boolean conditions within your document.

  \item \textbf{Lua Code Organization:} While you can embed Lua code directly with \verb|\directlua|, externalizing the code offers several advantages:
  \begin{itemize}
    \item Better syntax highlighting and code presentation in editors that support \tkzname{Lua} and \LATEX{}.
    \item Simplified commenting: \tkzname{Lua} uses \verb|--|, while \LATEX{} uses \verb|%|. Keeping Lua in a separate file avoids confusion.
    \item Reusability: external code files can be reused across multiple documents or figures.
  \end{itemize}
\end{itemize}

For simplicity, this documentation uses embedded Lua code in most cases. However, in some examples, external files are used to show you how it's done.


\vspace{12pt}
\begin{minipage}{.55\textwidth}
\begin{verbatim}
% !TEX TS-program = lualatex
% Created by Alain Matthes on 2024-01-09.
\documentclass[margin = 12pt]{standalone} 
\usepackage[mini]{tkz-euclide}
\usepackage{tkz-elements,ifthen}

\begin{document}  
\directlua{
 init_elements() % Clear tables
 dofile ("sangaku.lua") % Load the lua code 
}
\begin{tikzpicture}[ scale = .75]
 \tkzGetNodes
 \tkzDrawCircle(I,F)
 \tkzFillPolygon[color = purple](A,C,D)
 \tkzFillPolygon[color = blue!50!black](A,B,C)
 \tkzFillCircle[color = orange](I,F)
\end{tikzpicture}
\end{document}
\end{verbatim}
\end{minipage}
\begin{minipage}{.45\textwidth}
\directlua{
 init_elements()
 dofile ("sangaku.lua")
}

\begin{center}
\begin{tikzpicture}[ scale = .75]
  \tkzGetNodes
  \tkzDrawCircle(I,F)
  \tkzFillPolygon[color = purple](A,C,D)
  \tkzFillPolygon[color = blue!50!black](A,B,C)
  \tkzFillCircle[color = orange](I,F)
\end{tikzpicture}
\end{center}
\end{minipage}

\vspace{12pt}
And here is the code for the \tkzname{Lua} part: the file |ex_sangaku.lua|

\vspace{1em}
\begin{minipage}{.5\textwidth}
\begin{mybox}
\begin{verbatim}
 z.A = point(0, 0)
 z.B = point(8, 0)
 L.AB = line(z.A, z.B)
 S.AB = L.AB:square()
 _, _, z.C, z.D = S.AB:get()
 z.F = S.ac:projection(z.B)
 L.BF = line(z.B, z.F)
 T.ABC = triangle(z.A, z.B, z.C)
 L.bi = T.ABC:bisector(2)
 z.c = L.bi.pb
 L.Cc = line(z.C, z.c)
 z.I = intersection(L.Cc, L.BF)
\end{verbatim}
\end{mybox}
\end{minipage}

\subsection{Scaling Policy Update} % (fold)
\label{sub:scale_problem}

In previous versions, it was recommended to apply scaling within the Lua part of the code. However, this guidance has now changed.

Since all geometric computations are handled in \tkzname{Lua}, applying scaling in \TIKZ{} no longer presents any issues. On the contrary, performing scaling in \tkzname{Lua} has led to several complications—particularly with the recent implementation of conic-related functions, which involve numerous distance calculations using real numbers.

These challenges prompted a review of several functions, during which some bugs related to Lua-side scaling were identified and resolved.

\medskip
\noindent
\textbf{\textcolor{red}{New Recommendation:}} From now on, scaling should be applied exclusively in the \TIKZ{} part. The \tkzname{Lua} code should operate in an unscaled, consistent coordinate system to ensure the reliability of all geometric computations.

The following documentation uses only scaling in the \tkzEnv{tikz}{tikzpicture} environment.

% subsection scale_problem (end)
% section work_organization (end)