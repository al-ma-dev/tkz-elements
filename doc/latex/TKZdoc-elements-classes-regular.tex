\newpage
\section{Class \tkzClass{regular polygon}}

The variable \tkzVar{regular\_polygon}{RP} holds a table used to store regular polygons. It is optional, and you are free to choose the variable name. However, using \code{RP} is a recommended convention for clarity and consistency. If you use a custom variable (e.g., REP), you must initialize it manually. The \tkzFct{tkz-elements}{init\_elements()} function reinitializes the \code{RP} table if used.


\subsection{Creating a regular polygon}

The \tkzClass{regular\_polygon} class builds a regular polygon from its center, a first vertex, and the number of sides.

\medskip
The result is usually stored in \tkzVar{regular\_polygon}{P}, a variable used to group polygonal objects.

\begin{mybox}
\begin{verbatim}
P.hex = regular_polygon:new(z.O, z.A, 6)
\end{verbatim}
\end{mybox}

\textbf{Short form.}

Use the short form for brevity:

\begin{mybox}
\begin{verbatim}
P.hex = regular_polygon(z.O, z.A, 6)
\end{verbatim}
\end{mybox}

\subsection{Regular\_polygon attributes}
\label{sub:regular_polygon_attributes}

\begin{mybox}
Creation | RP.IA = regular_polygon(z.I,z.A,6)|
\end{mybox}


  \bgroup
  \catcode`_=12
  \small
  \captionof{table}{Regular\_polygon attributes.}\label{regular:attributes}
  \begin{tabular}{ll}
  \toprule
  \textbf{Attributes}      & \textbf{Application}  \\
 \midrule
  \tkzAttr{regular}{center}  & |z.I = RP.IA.center|  \\
  \tkzAttr{regular}{vertices}   & array containing all vertex affixes  \\
  \tkzAttr{regular}{through} & first vertex  \\
  \tkzAttr{regular}{circle}  & defines the circle with center I passing through A \\
  \tkzAttr{regular}{type}    & |RP.IA.type= 'regular\_polygon'|   \\
  \tkzAttr{regular}{side}    & |s = RP.IA.side| ; s = length of side\\
  \tkzAttr{regular}{circumradius}&  |S.AB.circumradius| ; radius of the circumscribed circle \\
  \tkzAttr{regular}{inradius}&  |S.AB.inxradius| ; radius of the inscribed circle   \\
  \tkzAttr{regular}{proj}    &  |RP.IA.proj| ; projection of the center on one side   \\
  \tkzAttr{regular}{angle}   &  |RP.IA.angle| ; angle formed by the center and 2 consecutive vertices   \\
  \bottomrule %
  \end{tabular}
  \egroup




\subsubsection{Pentagon}
\label{ssub:pentagon}
\begin{minipage}{.5\textwidth}
\begin{verbatim}
\directlua{
init_elements()
z.O = point(0, 0)
z.I = point(1, 3)
z.A = point(2, 0)
RP.five  = regular_polygon(z.I,z.A,5)
RP.five : name ("P_")
C.ins  = circle:radius(z.I,
             RP.five.inradius)
z.H = RP.five.proj
}
\begin{tikzpicture}[scale = .75]
\def\nb{\tkzUseLua{RP.five.nb}}
\tkzGetNodes
\tkzDrawCircles(I,A I,H)
\tkzDrawPolygon(P_1,P_...,P_\nb)
\tkzDrawPoints[red](P_1,P_...,P_\nb,H,I)
\tkzLabelPoints[red](I,A,H)
\end{tikzpicture}
\end{verbatim}
\end{minipage}
\begin{minipage}{.5\textwidth}
\directlua{
   init_elements()
   z.O = point(0, 0)
   z.I = point(1, 3)
   z.A = point(2, 0)
   RP.five = regular_polygon(z.I,z.A,5)
   RP.five : name ("P_")
   C.ins  = circle:radius( z.I , RP.five.inradius )
   z.H = RP.five.proj
  }

\begin{center}
  \begin{tikzpicture}[scale = .75]
  \def\nb{\tkzUseLua{RP.five.nb}}
  \tkzGetNodes
  \tkzDrawCircles(I,A I,H)
  \tkzDrawPolygon(P_1,P_...,P_\nb)
  \tkzDrawPoints[red](P_1,P_...,P_\nb,H,I)
  \tkzLabelPoints[red](I,A,H)
  \end{tikzpicture}
\end{center}


\end{minipage}

\subsection{Regular\_polygon methods}
\label{sub:regular_polygon_methods}


  \bgroup
  \catcode`_=12
  \small
  \captionof{table}{regular\_polygon methods.}\label{regular:methods}
  \begin{tabular}{ll}
  \toprule
  \textbf{Functions and Methods} & \textbf{Comments}    \\
  \midrule
  \tkzFct{regular\_polygon}{new(O,A,n)} & [\ref{ssub:function_regular_new}] \\
  \midrule
  \textbf{Circle} &\\
  \midrule
  \tkzMeth{regular\_polygon}{incircle ()} & [\ref{ssub:method_regular_incirle}]\\
  \midrule
  \textbf{Points} &\\
  \midrule
  \tkzMeth{regular\_polygon}{name (string)} & [\ref{ssub:method_regular_name}] \\
  \bottomrule %
  \end{tabular}
  \egroup


\subsubsection{Method \tkzMeth{regular}{new(pt, pt, n)}}
\label{ssub:function_regular_new}

\begin{tkzexample}[latex=.5\textwidth]
  \directlua{
   z.A = point(0, -4)
   z.O = point(0, 0)
   RP.six = regular_polygon(z.O, z.A, 6)
   RP.six:name("A_")
   z.i, z.p = RP.six:incircle():get()}

  \begin{tikzpicture}[gridded,scale=.75]
    \tkzGetNodes
    \tkzDrawCircles[red](O,A)
    \tkzDrawCircles[teal](i,p)
    \tkzDrawPolygon(A_1,A_...,A_6)
    \tkzDrawPoints[red](A_1,A_...,A_6)
    \tkzLabelPoints[red](A_1,A_...,A_6)
  \end{tikzpicture}
\end{tkzexample}

\subsubsection{Method \tkzMeth{regular}{incirle()}}
\label{ssub:method_regular_incirle}

See  previous example [\ref{ssub:function_regular_new}]

\subsubsection{Method \tkzMeth{regular}{name(s)}}
\label{ssub:method_regular_name}

See   [\ref{ssub:function_regular_new}]
\endinput