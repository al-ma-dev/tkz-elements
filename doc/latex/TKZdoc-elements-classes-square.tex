\newpage
\section{Class \tkzClass{square}}
The variable \tkzVar{square}{S} holds a table used to store squares. It is optional, and you are free to choose the variable name. However, using \code{S} is a recommended convention for clarity and consistency. If you use a custom variable (e.g., Squares), you must initialize it manually. The \code{init\_elements()} function reinitializes the \code{S} table if used.


\subsection{Creating a square}
\label{sub:creating_a_square}

The \tkzClass{square} class constructs a square from two adjacent vertices. The order of the points defines the orientation.


\medskip
The result is stored in \tkzVar{square}{S}.

\begin{mybox}
\begin{verbatim}
S.ABCD = square:new(z.A, z.B)
\end{verbatim}
\end{mybox}

\paragraph{Short form.}
The short form \code{square(z.A, z.B)} is equivalent:

\begin{mybox}
\begin{verbatim}
S.ABCD = square(z.A, z.B)
\end{verbatim}
\end{mybox}

\subsection{Square attributes}
\label{sub:square_attributes}
Points are created in the direct direction. A test is performed to check whether the points form a square. Otherwise, compilation is blocked."
\begin{mybox}
Creation | S.AB = square(z.A,z.B,z.C,z.D)|
\end{mybox}

\begin{center}
  \bgroup
  \catcode`_=12
  \small
  \captionof{table}{Square attributes.}\label{square:att}
  \begin{tabular}{lll}
  \toprule
  \textbf{Attributes}        & \textbf{Application}  &  \\
  \midrule
  \tkzAttr{square}{pa}         & |z.A = S.AB.pa|       &  \\
  \tkzAttr{square}{pb}         & |z.B = S.AB.pb|       &  \\
  \tkzAttr{square}{pc}         & |z.C = S.AB.pc|       &  \\
  \tkzAttr{square}{pd}         & |z.D = S.AB.pd|       &  \\
  \tkzAttr{square}{type}       & |S.AB.type= 'square'| &  \\
  \tkzAttr{square}{side}       & |s = S.AB.center|     & s = length of side   \\
  \tkzAttr{square}{center}     & |z.I = S.AB.center|   & center of the square \\
  \tkzAttr{square}{circumradius}   & |S.AB.circumradius|       & radius of the circumscribed circle \\
  \tkzAttr{square}{inradius}   & |S.AB.inxradius|      & radius of the inscribed circle   \\
  \tkzAttr{square}{proj}       & |S.AB.proj|           & projection of the center on one side \\
  \tkzAttr{square}{ab}         & |S.AB.ab|             &  line passing through two vertices   \\
  \tkzAttr{square}{ac}         & |S.AB.ca|             &  idem. \\
  \tkzAttr{square}{ad}         & |S.AB.ad|             &  idem. \\
  \tkzAttr{square}{bc}         & |S.AB.bc|             &  idem. \\
  \tkzAttr{square}{bd}         & |S.AB.bd|             &  idem. \\
  \tkzAttr{square}{cd}         & |S.AB.cd|             &  idem. \\
  \bottomrule %
  \end{tabular}
  \egroup
\end{center}




\subsubsection{Example: square attributes }
\label{ssub:example_square_attributes}

\begin{minipage}{.5\textwidth}
\begin{verbatim}
\directlua{
 init_elements()
 z.A = point(0, 0)
 z.B = point(4, 0)
 z.C = point(4, 4)
 z.D = point(0, 4)
 S.new = square(z.A, z.B, z.C, z.D)
 z.I = S.new.center
 z.H = S.new.proj}
\begin{tikzpicture}
\tkzGetNodes
\tkzDrawCircles[orange](I,A I,H)
\tkzDrawPolygon(A,B,C,D)
\tkzDrawPoints(A,B,C,D,H,I)
\tkzLabelPoints(A,B,H,I)
\tkzLabelPoints[above](C,D)
\tkzDrawSegments(I,B I,H)
\tkzLabelSegment[sloped](I,B){%
  \pmpn{\tkzUseLua{S.new.circumradius}}}
\tkzLabelSegment[sloped](I,H){%
  \pmpn{\tkzUseLua{S.new.inradius}}}
\tkzLabelSegment[sloped](D,C){%
  \pmpn{\tkzUseLua{S.new.side}}}
\end{tikzpicture}
\end{verbatim}
\end{minipage}
\begin{minipage}{.5\textwidth}
\directlua{
 init_elements()
 z.A = point(0, 0)
 z.B = point(4, 0)
 z.C = point(4, 4)
 z.D = point(0, 4)
 S.new = square(z.A, z.B, z.C, z.D)
 z.I = S.new.center
 z.H = S.new.proj}

\begin{center}
  \begin{tikzpicture}
  \tkzGetNodes
  \tkzDrawCircles[orange](I,A I,H)
  \tkzDrawPolygon(A,B,C,D)
  \tkzDrawPoints(A,B,C,D,H,I)
  \tkzLabelPoints(A,B,H,I)
  \tkzLabelPoints[above](C,D)
  \tkzDrawSegments(I,B I,H)
  \tkzLabelSegment[sloped](I,B){\pmpn{\tkzUseLua{S.new.circumradius}}}
  \tkzLabelSegment[sloped](I,H){\pmpn{\tkzUseLua{S.new.inradius}}}
  \tkzLabelSegment[sloped](D,C){\pmpn{\tkzUseLua{S.new.side}}}
  \end{tikzpicture}
\end{center}
\end{minipage}


\newpage


\subsection{Square Methods}
\label{sub:square_methods}

\begin{center}
  \bgroup
  \catcode`_=12
  \small
  \captionof{table}{Square methods}\label{square:met}
  \begin{tabular}{lll}
  \toprule
  \textbf{Methods} & \textbf{Reference}&    \\
  \midrule

  \tkzMeth{square}{new(za,zb,zc,zd)} &Note\footnote{square(pt,pt,pt,pt) (short form, recommended)};[\ref{ssub:method_square_rotation}]\\

  \tkzMeth{square}{rotation (zi,za)} &[\ref{ssub:method_square_rotation}]\\

  \tkzMeth{square}{side (za,zb,"swap")} & \ref{ssub:square_with_side_method} \\
  \bottomrule %
  \end{tabular}
  \egroup
\end{center}



\subsubsection{Method \tkzFct{square}{rotation(pt,pt)}}
\label{ssub:method_square_rotation}
$I$ square center; $A$ first vertex

\vspace{1em}
\begin{tkzexample}[latex=.5\textwidth]
\directlua{
 z.A  = point(0, 0)
 z.I  = point(2, -1)
 S    = square:rotation(z.I, z.A)
 z.B  = S.pb
 z.C  = S.pc
 z.D  = S.pd
 z.I  = S.center}
\begin{tikzpicture}
\tkzGetNodes
\tkzDrawPolygon(A,B,C,D)
\tkzDrawPoints(A,B,C,D)
\tkzLabelPoints(B)
\tkzLabelPoints[above](A)
\tkzLabelPoints[right](C,D)
\tkzDrawPoints[red](I)
\end{tikzpicture}
\end{tkzexample}

\subsubsection{Method \tkzFct{square}{side (za,zb)}}
\label{ssub:square_with_side_method}
With the option \code{"swap"} then the square is defined in counterclockwise. The result can also be obtained from a line  [\ref{ssub:method_line_square}].

\begin{minipage}{.5\textwidth}
\begin{verbatim}
\directlua{
 init_elements()
 z.A = point(0, 0)
 z.B = point(2, 1)
 S.side = square:side(z.A, z.B)
 z.B = S.side.pb
 z.C = S.side.pc
 z.D = S.side.pd
 z.I = S.side.center}
\begin{tikzpicture}[scale = 2]
   \tkzGetNodes
   \tkzDrawPolygon(A,B,C,D)
   \tkzDrawPoints(A,B,C,D)
   \tkzLabelPoints(A,B)
   \tkzLabelPoints[above](C,D)
   \tkzDrawPoints[red](I)
\end{tikzpicture}
\end{verbatim}
\end{minipage}
\begin{minipage}{.5\textwidth}
\directlua{
 init_elements()
 z.A = point(0, 0)
 z.B = point(2, 1)
 S.side = square:side(z.A, z.B)
 z.B = S.side.pb
 z.C = S.side.pc
 z.D = S.side.pd
 z.I = S.side.center}
\begin{center}
  \begin{tikzpicture}[scale = 2]
  \tkzGetNodes
  \tkzDrawPolygon(A,B,C,D)
  \tkzDrawPoints(A,B,C,D)
  \tkzLabelPoints(A,B)
  \tkzLabelPoints[above](C,D)
  \tkzDrawPoints[red](I)
  \end{tikzpicture}
\end{center}

\end{minipage}
\endinput