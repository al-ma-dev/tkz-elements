\newpage
\section{Transfers}

\subsection{From Lua to tkz-euclide or TikZ}

This section explains how to transfer values from Lua to TeX/TikZ, including points, booleans, numeric values, and functional data. Some examples are advanced and can be explored in a second reading.


\subsubsection{Points transfer}
\label{ssub:points_transfer}

The necessary definitions and calculations are performed with the primitive \tkzMacro{lualatex}{directlua}\footnote{Or inside the \tkzEnv{tkz-elements}{tkzelements} environment}. Then, we execute the macro \tkzcname{tkzGetNodes}, which transforms the affixes of the table |z| into TikZ nodes. The drawing can then be done with plain \TIKZ{} or \tkzNamePack{tkz-euclide}.

If you wish to use another package for plotting, you must define a custom macro similar to \tkzMacro{tkz-elements}{tkzGetNodes} that interprets the contents of the |z| table. You don't need to understand the following code to use the package. You only need to know that if |z.A| is defined, then the macro below will create a node named |A|.

\vspace*{1em}
\begin{mybox}
\begin{verbatim}
\def\tkzGetNodes{\directlua{
   for K,V in pairs(z) do
      local n,sd,ft
      n = string.len(K)
      if n >1 then
      _,_,ft, sd = string.find( K , "(.+)(.)" )
     if sd == "p" then   K=ft.."'" end
     _,_,xft, xsd = string.find( ft , "(.+)(.)" )
     if xsd == "p" then  K=xft.."'".."'" end
       end
  tex.print("\\coordinate ("..K..") at ("..V.re..","..V.im..") ;\\\\")
end}
}
\end{verbatim}
\end{mybox}

See the section In-depth Study \ref{sec:in_depth_study} for an explanation of the previous code.

Point names may include the underscore  |_| and the macro \tkzMacro{tkz-elements}{tkzGetNodes} for automatic conversion to names with \tkzname{prime} or \tkzname{double prime}. (See  the next example)

\vspace{1em}

\begin{tkzexample}[latex=7cm]
  \directlua{
     init_elements()
     z.o   = point(0, 0)
     z.a_1 = point(2, 1)
     z.a_2 = point(1, 2)
     z.ap  = z.a_1 + z.a_2
     z.app = z.a_1 - z.a_2
  }

  \begin{center}
    \begin{tikzpicture}[ scale = 1.5]
       \tkzGetNodes
       \tkzDrawSegments(o,a_1 o,a_2 o,a' o,a'')
       \tkzDrawSegments[red](a_1,a' a_2,a')
       \tkzDrawSegments[blue](a_1,a'' a_2,a'')
       \tkzDrawPoints(a_1,a_2,a',o,a'')
       \tkzLabelPoints(o,a_1,a_2,a',a'')
    \end{tikzpicture}
  \end{center}
\end{tkzexample}

\subsubsection{Other transfers}

It is also useful to transfer numerical values (such as lengths or angles) and booleans. For this, we provide the macro \tkzMacro{tkz-elements}{tkzUseLua(...)}:

\begin{mybox}
  \begin{verbatim}
  \def\tkzUseLua#1{\directlua{tex.print(tostring(#1))}}
\end{verbatim}
\end{mybox}

This macro prints the value of a Lua variable or expression directly into the TeX stream.

\paragraph{Example.} The following Lua code computes whether two lines intersect:

\directlua{
 z.a = point(4, 2)
 z.b = point(1, 1)
 z.c = point(2, 2)
 z.d = point(5, 1)
 L.ab = line(z.a,z.b)
 L.cd = line(z.c,z.d)
 det = (z.b-z.a)^(z.d-z.c)
 if det == 0 then bool = true
   else bool = false
 end
 x = intersection (L.ab,L.cd)}

The intersection of the two lines lies at
a point whose affix is: \tkzUseLua{x}

\vspace{1em}
\begin{minipage}{0.5\textwidth}
\begin{verbatim}
\directlua{
  z.a = point(4, 2)
  z.b = point(1, 1)
  z.c = point(2, 2)
  z.d = point(5, 1)
  L.ab = line(z.a, z.b)
  L.cd = line(z.c, z.d)
  det = (z.b - z.a)^(z.d - z.c)
  if det == 0 then bool = true
    else bool = false
  end
  x = intersection (L.ab,L.cd)}

The intersection of the two lines lies at
    a point whose affix is:\tkzUseLua{x}


\begin{tikzpicture}
  \tkzGetNodes
  \tkzInit[xmin =0,ymin=0,xmax=5,ymax=3]
  \tkzGrid\tkzAxeX\tkzAxeY
  \tkzDrawPoints(a,...,d)
  \ifthenelse{\equal{\tkzUseLua{bool}}{%
  true}}{\tkzDrawSegments[red](a,b c,d)}{%
  \tkzDrawSegments[blue](a,b c,d)}
  \tkzLabelPoints(a,...,d)
\end{tikzpicture}
\end{verbatim}
\end{minipage}
\begin{minipage}{0.5\textwidth}
\begin{center}
\begin{tikzpicture}
  \tkzGetNodes
  \tkzInit[xmin =0,ymin=0,xmax=5,ymax=3]
  \tkzGrid\tkzAxeX\tkzAxeY
  \tkzDrawPoints(a,...,d)
  \ifthenelse{\equal{\tkzUseLua{bool}}{true}}{
  \tkzDrawSegments[red](a,b c,d)}{%
  \tkzDrawSegments[blue](a,b c,d)}
  \tkzLabelPoints(a,...,d)
\end{tikzpicture}
\end{center}
  \end{minipage}

\subsubsection{Example 1}
\label{ssub:example_1}

This example shows how to transfer a Lua-defined function and generate a path of coordinates to be used in a TikZ plot. The main tools involved are Lua's   function \tkzFct{lua}{load}  and the   \tkzNameObj{path} class. See [\ref{ssub:plotting_a_curve}]


\begin{verbatim}
\makeatletter\let\percentchar\@percentchar\makeatother
\directlua{
init_elements()
function list (f,min,max,nb)
  PA.tbl = path()
  for x = min, max, (max - min) / nb do
     PA.tbl:add_point(point(x, f(x)),5)
  end
  return table.concat(PA.tbl)
end}
\def\plotcoords#1#2#3#4{%
\directlua{%
  f = load (([[
        return function (x)
            return (\percentchar s)
        end
    ]]):format ([[#1]]), nil, 't', math) ()
tex.print(list(f,#2,#3,#4))}}

\begin{tikzpicture}
\tkzInit[xmin=1,xmax=3,ymin=0,ymax=2]
\tkzGrid
\tkzDrawX[right=3pt,label={$x$}]
\tkzDrawY[above=3pt,label={$f(x) = \dfrac{1-\mathrm{e}^{-x^2}}{1+\mathrm{e}^{-x^2}}$}]
\draw[cyan,thick] plot coordinates {\plotcoords{(1-exp(-x^2))/(exp(-x^2)+1)}{-3}{3}{100}};
\end{tikzpicture}
\end{verbatim}


\makeatletter\let\percentchar\@percentchar\makeatother
\directlua{
init_elements()
function list (f,min,max,nb)
  PA.tbl = path()
  for x = min, max, (max - min) / nb do
     PA.tbl:add_point(point(x, f(x)),5)
  end
  return table.concat(PA.tbl)
end}
\def\plotcoords#1#2#3#4{%
\directlua{%
  f = load (([[
        return function (x)
            return (\percentchar s)
        end
    ]]):format ([[#1]]), nil, 't', math) ()
tex.print(list(f,#2,#3,#4))}
}
\begin{center}
  \begin{tikzpicture}
  \tkzInit[xmin=1,xmax=3,ymin=0,ymax=2]
  \tkzGrid
  \tkzDrawX[right=3pt,label={$x$}]
  \tkzDrawY[above=3pt,label={$f(x) = \dfrac{1-\mathrm{e}^{-x^2}}{1+\mathrm{e}^{-x^2}}$}]
  \draw[cyan,thick] plot coordinates {\plotcoords{(1-exp(-x^2))/(exp(-x^2)+1)}{-3}{3}{100}};
  \end{tikzpicture}
\end{center}

\subsubsection{Example 2}

This example demonstrates how to pass a value (the number of sides) from \TeX{} to Lua using the   \tkzMacro{lualatex}{directlua} primitive. This enables the dynamic creation of regular polygons. This example is based on a answer from egreg.
\begin{flushright}
\small
 \href{https://tex.stackexchange.com/questions/729009/how-can-these-regular-polygons-be-arranged-within-a-page/731503#731503}{egreg--tex.stackexchange.com}
\end{flushright}

\begin{verbatim}
\directlua{
  z.I = point(0, 0)
  z.A = point(2, 0)
}
\def\drawPolygon#1{
\directlua{
  RP.six = regular_polygon(z.I,z.A,#1)
  RP.six : name ("P_")
  }
\begin{tikzpicture}[scale=.5]
 \def\nb{\tkzUseLua{RP.six.nb}}
 \tkzGetNodes
 \tkzDrawCircles(I,A)
 \tkzDrawPolygon(P_1,P_...,P_\nb)
 \tkzDrawPoints[red](P_1,P_...,P_\nb)
\end{tikzpicture}
}
\foreach [count=\i] \n in {3, 4, ..., 10} {
  \makebox[0.2\textwidth]{%
    \begin{tabular}[t]{@{}c@{}}
      $n=\n$ \\[1ex]
      \drawPolygon{\n}
    \end{tabular}%
  }\ifnum\i=4 \\[2ex]\fi
}
\end{verbatim}

\directlua{
  z.I = point(0,0)
  z.A = point(2,0)
}
\def\drawPolygon#1{
\directlua{
  RP.six   = regular_polygon(z.I,z.A,#1)
  RP.six : name ("P_")
  }
\begin{tikzpicture}[scale=.5]
 \def\nb{\tkzUseLua{RP.six.nb}}
 \tkzGetNodes
 \tkzDrawCircles(I,A)
 \tkzDrawPolygon(P_1,P_...,P_\nb)
 \tkzDrawPoints[red](P_1,P_...,P_\nb)
\end{tikzpicture}
}
\foreach [count=\i] \n in {3, 4, ..., 10} {
  \makebox[0.2\textwidth]{%
    \begin{tabular}[t]{@{}c@{}}
      $n=\n$ \\[1ex]
      \drawPolygon{\n}
    \end{tabular}%
  }\ifnum\i=4 \\[2ex]\fi
}

\subsubsection{Example 3}

This time, the transfer will be carried out using an external file. The following example is based on this one, but using a table.

\directlua{
 init_elements()
   z.a = point(1, 0)
   z.b = point(3, 2)
   z.c = point(0, 2)
 A,B,C =  tkz.parabola (z.a, z.b, z.c)

 function f(t0, t1, n)
  local out=assert(io.open("tmp.table","w"))
  local y
  for t = t0,t1,(t1-t0)/n  do
   y = A*t^2+B*t +C
   out:write(utils.checknumber(t), " ",
     utils.checknumber(y), " i\string\n")
  end
  out:close()
 end
 }

\begin{minipage}{0.55\textwidth}
\begin{verbatim}
\directlua{
  init_elements()
   z.a = point(1, 0)
   z.b = point(3, 2)
   z.c = point(0, 2)
   A,B,C = parabola (z.a, z.b, z.c)

 function f(t0, t1, n)
  local out=assert(io.open("tmp.table","w"))
  local y
  for t = t0,t1,(t1-t0)/n  do
   y = A*t^2+B*t +C
   out:write(utils.checknumber(t), " ",
     utils.checknumber(y), " i\string\n")
  end
  out:close()
 end
 }
\begin{tikzpicture}
   \tkzGetNodes
   \tkzInit[xmin=-1,xmax=5,ymin=0,ymax=5]
   \tkzDrawX\tkzDrawY
   \tkzDrawPoints[red,size=2](a,b,c)
   \directlua{f(-1,3,100)}%
   \draw[domain=-1:3] plot[smooth]
        file {tmp.table};
\end{tikzpicture}
\end{verbatim}
\end{minipage}
\begin{minipage}{0.45\textwidth}
\begin{center}
  \begin{tikzpicture}
     \tkzGetNodes
     \tkzInit[xmin=-1,xmax=5,ymin=0,ymax=5]
     \tkzDrawX\tkzDrawY
     \tkzDrawPoints[red,size=2](a,b,c)
     \directlua{f(-1,3,100)}%
     \draw[domain=-1:3] plot[smooth] file {tmp.table};
  \end{tikzpicture}
\end{center}
\end{minipage}

\subsubsection{Example 4}

The result is identical to the previous one.
\begin{verbatim}
\directlua{
   z.a   = point(1, 0)
   z.b   = point(3, 2)
   z.c   = point(0, 2)
   A,B,C =  parabola (z.a, z.b, z.c)

 function f(t0, t1, n)
 local PA.tbl = path()
 for t = t0,t1,(t1-t0)/n  do
     y = A*t^2+B*t +C
     local pt = point(t, y)
     PA.tbl:add_point(pt)
  end
  return table.concat (tbl)
end
}
\begin{tikzpicture}
   \tkzGetNodes
   \tkzDrawX\tkzDrawY
   \tkzDrawPoints[red,size=2pt](a,b,c)
   \draw[domain=-2:3,smooth] plot coordinates {\directlua{tex.print(f(-2,3,100))}};
\end{tikzpicture}
\end{verbatim}

\subsubsection{Example 5}

\begin{verbatim}
\makeatletter\let\percentchar\@percentchar\makeatother
\directlua{
function cellx (start,step,n)
return start+step*(n-1)
end
}
\def\calcval#1#2{%
\directlua{
  f = load (([[
        return function (x)
            return (\percentchar s)
        end
    ]]):format ([[#1]]), nil, 't', math) ()
x = #2
tex.print(string.format("\percentchar.2f",f(x)))}
}
\def\fvalues(#1,#2,#3,#4) {%
\def\firstline{$x$}
    \foreach \i in {1,2,...,#4}{%
      \xdef\firstline{\firstline &  \tkzUseLua{cellx(#2,#3,\i)}}}
\def\secondline{$f(x)=#1$}
     \foreach \i in {1,2,...,#4}{%
      \xdef\secondline{\secondline &
     \calcval{#1}{\tkzUseLua{cellx(#2,#3,\i)}}}}
\begin{tabular}{l*{#4}c}
  \toprule
  \firstline  \\
  \secondline \\
  \bottomrule
  \end{tabular}
}
\fvalues(x^2-3*x+1,-2,.25,8)
\vspace{12pt}

\end{verbatim}

\makeatletter\let\percentchar\@percentchar\makeatother
\directlua{
function cellx (start,step,n)
return start+step*(n-1)
end
}
\def\calcval#1#2{%
\directlua{
  f = load (([[
        return function (x)
            return (\percentchar s)
        end
    ]]):format ([[#1]]), nil, 't', math) ()
x = #2
tex.print(string.format("\percentchar.2f",f(x)))}
}
\def\fvalues(#1,#2,#3,#4) {%
\def\firstline{$x$}
    \foreach \i in {1,2,...,#4}{%
      \xdef\firstline{\firstline &  \tkzUseLua{cellx(#2,#3,\i)}}}
\def\secondline{$f(x)=#1$}
     \foreach \i in {1,2,...,#4}{%
      \xdef\secondline{\secondline &
     \calcval{#1}{\tkzUseLua{cellx(#2,#3,\i)}}}}
\begin{tabular}{l*{#4}c}
  \toprule
  \firstline  \\
  \secondline \\
  \bottomrule
  \end{tabular}
}
\fvalues(x^2-3*x+1,-2,.25,8)

\subsubsection{Summary}
The transfer of data between Lua and TeX is a key feature of \tkzNamePack{tkz-elements}, enabling high-precision numerical computations and dynamic figure generation. Whether for plotting curves, testing geometric properties, or generating tables, these tools offer flexibility and power that extend beyond traditional TeX capabilities.
\endinput