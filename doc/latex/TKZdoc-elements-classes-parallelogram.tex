\newpage
\section{Class \tkzClass{parallelogram}}
The variable \tkzVar{parallelogram}{P} holds a table used to store parallelograms. It is optional, and you are free to choose the variable name. However, using \code{P} is a recommended convention for clarity and consistency. If you use a custom variable (e.g., parall), you must initialize it manually. The \code{init\_elements()} function reinitializes the \code{P} table if used.


\subsection{Creating a parallelogram}
\label{sub:creating_a_parallelogram}

The \tkzClass{parallelogram} class creates a parallelogram using three points. The fourth vertex is computed automatically.

\medskip
The resulting object is stored in \tkzVar{parallelogram}{P}. You are free to use another name, but \tkzVar{parallelogram}{P} is preferred for consistency.

\begin{mybox}
\begin{verbatim}
P.ABCD = parallelogram:new(z.A, z.B, z.C)
\end{verbatim}
\end{mybox}

\paragraph{Short form.}
You may also use the short form:

\begin{mybox}
\begin{verbatim}
P.ABCD = parallelogram(z.A, z.B, z.C)
\end{verbatim}
\end{mybox}

\subsection{Parallelogram attributes}

Points are created in the direct direction. A test is performed to check whether the points form a parallelogram, otherwise compilation is blocked.

\begin{mybox}
Creation | P.new = parallelogram(z.A,z.B,z.C,z.D)|
\end{mybox}

\begin{center}
  \bgroup
  \catcode`_=12
  \small
  \captionof{table}{Parallelogram attributes.}\label{parallelogram:att}
  \begin{tabular}{lll}
  \toprule
  \textbf{Attributes}         & \textbf{Application}  &  \\
  \midrule
  \tkzAttr{parallelogram}{pa}   & |z.A = P.new.pa|     &  \\
  \tkzAttr{parallelogram}{pb}   & |z.B = P.new.pb|     &  \\
  \tkzAttr{parallelogram}{pc}   & |z.C = P.new.pc|     &  \\
  \tkzAttr{parallelogram}{pd}   & |z.D = P.new.pd|     &  \\
  \tkzAttr{parallelogram}{type} & |P.new.type= 'parallelogram'|&  \\
  \tkzAttr{parallelogram}{i}    & |z.I = P.new.i|  & intersection of diagonals \\
  \tkzAttr{parallelogram}{ab}   & |P.new.ab|    &  line passing through two vertices \\
  \tkzAttr{parallelogram}{ac}   & |P.new.ca|           &  idem. \\
  \tkzAttr{parallelogram}{ad}   & |P.new.ad|           &  idem. \\
  \tkzAttr{parallelogram}{bc}   & |P.new.bc|           &  idem. \\
  \tkzAttr{parallelogram}{bd}   & |P.new.bd|           &  idem. \\
  \tkzAttr{parallelogram}{cd}   & |P.new.cd|           &  idem. \\
  \bottomrule %
  \end{tabular}
  \egroup
\end{center}




\subsubsection{Example: attributes }
\label{ssub:example_attributes}
\begin{minipage}{.5\textwidth}
\begin{verbatim}
\directlua{
 init_elements()
 z.A = point(0, 0)
 z.B = point(4, 1)
 z.C = point(7, 5)
 z.D = point(3, 4)
 P.new = parallelogram(z.A, z.B, z.C, z.D)
 z.B = P.new.pb
 z.C = P.new.pc
 z.D = P.new.pd
 z.I = P.new.center}
\begin{tikzpicture}
\tkzGetNodes
\tkzDrawPolygon(A,B,C,D)
\tkzDrawPoints(A,B,C,D)
\tkzLabelPoints(A,B)
\tkzLabelPoints[above](C,D)
\tkzDrawPoints[red](I)
\end{tikzpicture}
\end{verbatim}
\end{minipage}
\begin{minipage}{.5\textwidth}
\directlua{
 init_elements()
 z.A = point(0, 0)
 z.B = point(4, 1)
 z.C = point(7, 5)
 z.D = point(3, 4)
 P.new = parallelogram(z.A, z.B, z.C, z.D)
 z.B = P.new.pb
 z.C = P.new.pc
 z.D = P.new.pd
 z.I = P.new.center}

\begin{center}
  \begin{tikzpicture}
  \tkzGetNodes
  \tkzDrawPolygon(A,B,C,D)
  \tkzDrawPoints(A,B,C,D)
  \tkzLabelPoints(A,B)
  \tkzLabelPoints[above](C,D)
  \tkzDrawPoints[red](I)
  \end{tikzpicture}
\end{center}

\end{minipage}


\newpage


\subsection{Parallelogram functions}
\label{sub:parallelogram_methods}
\begin{center}
  \bgroup
  \catcode`_=12
  \small
  \captionof{table}{Parallelogram functions.}\label{parallelogram:met}
  \begin{tabular}{ll}
  \toprule
  \textbf{Functions} & \textbf{Reference}    \\
  \midrule
  \tkzMeth{parallelogram}{new (za, zb, zc, zd)} & \ref{ssub:example_attributes}\\
  \tkzMeth{parallelogram}{fourth (za,zb,zc)} & \ref{ssub:parallelogram_with_fourth_method} \\
  \bottomrule %
  \end{tabular}
  \egroup
\end{center}



\subsubsection{Method \tkzMeth{parallelogram}{new(pt,pt,pt,pt)}}
\label{ssub:function_parallelogram_new}

\begin{tkzexample}[latex=.5\textwidth]
\directlua{
 z.A = point(0, 0)
 z.B = point(4, 1)
 z.C = point(7, 5)
 z.D = point(3, 4)
 P.ABCD = parallelogram(z.A, z.B, z.C, z.D)
 z.B = P.ABCD.pb
 z.C = P.ABCD.pc
 z.D = P.ABCD.pd
 z.I = P.ABCD.center}
\begin{tikzpicture}
\tkzGetNodes
\tkzDrawPolygon(A,B,C,D)
\tkzDrawPoints(A,B,C,D)
\tkzLabelPoints(A,B)
\tkzLabelPoints[above](C,D)
\tkzDrawPoints[red](I)
\end{tikzpicture}
\end{tkzexample}

\subsubsection{Method \tkzMeth{parallelogram}{fourth(pt,pt,pt)}}
\label{ssub:parallelogram_with_fourth_method}
completes a triangle by parallelogram (See  next example)

\begin{minipage}{.5\textwidth}
\begin{verbatim}
\directlua{
init_elements()
z.A = point(0, 0)
z.B = point(3, 1)
z.C = point(4, 3)
P.four= parallelogram:fourth(z.A, z.B, z.C)
z.D = P.four.pd
z.I = P.four.center}
\begin{tikzpicture}[ scale = .75]
\tkzGetNodes
\tkzDrawPolygon(A,B,C,D)
\tkzDrawPoints(A,B,C,D)
\tkzLabelPoints(A,B)
\tkzLabelPoints[above](C,D)
\tkzDrawPoints[red](I)
\end{tikzpicture}
\end{verbatim}
\end{minipage}
\begin{minipage}{.5\textwidth}
\directlua{
 init_elements()
 z.A = point(0, 0)
 z.B = point(3, 1)
 z.C = point(4, 3)
 P.four= parallelogram:fourth(z.A, z.B, z.C)
 z.D = P.four.pd
 z.I = P.four.center}
\begin{center}
  \begin{tikzpicture}[scale = .75]
  \tkzGetNodes
  \tkzDrawPolygon(A,B,C,D)
  \tkzDrawPoints(A,B,C,D)
  \tkzLabelPoints(A,B)
  \tkzLabelPoints[above](C,D)
  \tkzDrawPoints[red](I)
  \end{tikzpicture}
\end{center}
\end{minipage}
\endinput