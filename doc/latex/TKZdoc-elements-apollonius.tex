\newpage
\section{Apollonius' problem}
\label{sec:apollonius}

\subsection*{Overview}

The classical \texttt{Apollonius’ problem} asks for one or several circles tangent to three given geometric objects, each of which may be a circle (\tkzname{C}), a line (\tkzname{L}) or a point (\tkzname{P}).
Each case is denoted by a three-letter code such as \tkzname{CCP}, \tkzname{LLL}, or \tkzname{PLC}, according to the type of the three given objects.

Historically, there are ten distinct configurations:
\[
\text{PPP},\; \text{LPP},\; \text{CPP},\; \text{LLP},\; \text{LLL}
,\; \text{CLP},\;\text{CLL},\; \text{CCP},\; \text{CCL},\; \text{CCC}.
\]
Viète solved all ten cases using only compass and straightedge constructions, building complex solutions from simpler ones.

\medskip
\texttt{Note (work in progress): } This section is under active development. Some \emph{degenerate} or \emph{very special} configurations may not be fully handled yet.


\subsection*{Unified API (paths everywhere)}

Each case is implemented as a method attached to the class of the \emph{first} object (e.g. \verb|point:PPP|, \verb|line:LLL|, \verb|circle:CLL|).
All methods are being unified to return \emph{paths} so that drawing and node extraction are consistent across cases.

\medskip
\texttt{Syntax:} \verb|PA.center, PA.through, n = <obj>:XYZ(args)|

\medskip
\texttt{Returned values:}
\begin{itemize}
  \item \verb|PA.center| : a \texttt{path} collecting the centers of the solution circles;
  \item \verb|PA.through| : a \texttt{path} collecting the corresponding through-points;
  \item \verb|n| : the number of valid solutions (can be \verb|0|).
\end{itemize}

\medskip
\texttt{Remarks: }
\begin{itemize}
  \item All results can be directly drawn using
    \verb|\tkzDrawCirclesFromPaths(PA.center, PA.through)|. See [\ref{sub:macro_tkzDrawCirclesFromPaths}]
  \item Nodes can be exported to TikZ with
    \verb|tkz.nodes_from_paths(PA.center, PA.through)|. See [\ref{sub:tkz_nodes_from_paths}]
  \item Empty paths correspond to degenerate configurations.
  \item Some older methods still return \verb|paths, n|; they will be progressively updated to the unified format.
\end{itemize}

\subsection*{Using \tkzname{tkz.nodes\_from\_paths} instead of \textbackslash tkzDrawCirclesFromPaths}

If you wish to draw \emph{all} the solution circles at once, the macro
\verb|\tkzDrawCirclesFromPaths(PA.ce, PA.th)|
is the most convenient approach.

However, if you need to highlight or draw \emph{only one specific circle}, it is often preferable to use the Lua helper
\verb|tkz.nodes_from_paths|. In this case, you may keep the solutions in local variables to avoid polluting the global table:

\begin{verbatim}
local pc, pt, n = C.OC:CPP(z.A, z.B)
tkz.nodes_from_paths(pc, pt)   -- creates z.w1, z.t1, z.w2, z.t2,...
\end{verbatim}

By default, this generates nodes \verb|(w1,t1), (w2,t2), ...| in the table \verb|z|, which can then be used to draw individual circles or to label / style them differently.

\medskip
\texttt{Combining both approaches: }
You can also combine the two techniques:

first use \verb|\tkzDrawCirclesFromPaths| to display all solution circles, and then use \verb|tkz.nodes_from_paths| to extract the corresponding nodes and emphasize only some of them (change colour, add labels, draw radii, etc.). This is particularly useful in demonstrations or pedagogical figures.

\medskip
\texttt{Example idea:}
\begin{itemize}
  \item Compute the solutions: \verb|PA.center, PA.through, n = C.OC:CPP(z.A, z.B)|.
  \item Draw all circles with \verb|\tkzDrawCirclesFromPaths|.
  \item Call \verb|tkz.nodes_from_paths(PA.center, PA.through)|.
  \item Then, for instance: \verb|\draw[red,thick] (z.w1) circle[radius];| to highlight only the first one.
\end{itemize}

Thus, the macro gives a quick visual result, while \verb|tkz.nodes_from_paths| offers fine control over individual solutions.

\medskip
\texttt{Global \texttt{PA.xxx} vs Local Variables: }

When using contact methods such as:
\begin{verbatim}
PA.center, PA.through, n = C.OC:CPP(z.A, z.B)
\end{verbatim}
the two paths \verb|PA.center| and \verb|PA.through| are stored as \emph{global} variables in the
\texttt{z}/\texttt{PA} namespace.
This has two important advantages:

\begin{itemize}
  \item They can be used directly in TikZ with macros such as:

        \verb|\tkzDrawCirclesFromPaths(PA.center, PA.through)|.
  \item They are automatically cleared when calling \verb|init_elements()|, which avoids
        name conflicts between different figures or compilations.
\end{itemize}

Since TikZ macros work purely on the TeX side, the paths must exist globally to be accessible
outside the \verb|\directlua| block.

\medskip
\texttt{Using Local Paths with \texttt{tkz.nodes\_from\_paths}: }

If you do not want to pollute the global namespace, you may store the results locally:

\begin{verbatim}
local pc, pt, n = C.OC:CPP(z.A, z.B)
tkz.nodes_from_paths(pc, pt)  -- creates z.w1, z.t1,...
\end{verbatim}

Here, \verb|pc| and \verb|pt| remain Lua-local, but the function
\verb|tkz.nodes_from_paths| exports the corresponding nodes (\verb|w1,t1,w2,t2,...|)
into the global table \verb|z| so they can be used in TikZ.

\medskip
\texttt{Which One to Use?}

\begin{itemize}
  \item Use \verb|PA.center, PA.through| (global) if you plan to call:
        \begin{verbatim}
        \tkzDrawCirclesFromPaths(PA.center, PA.through)
        \end{verbatim}
        because this macro expects global paths accessible from TeX.
  \item Use \verb|tkz.nodes_from_paths| if you want to select or highlight only some of
        the solution circles. It also works with either global or local paths.
  \item A combined approach is often the most flexible: draw all circles with the macro,
        then export nodes to highlight specific ones.
\end{itemize}

\medskip
\texttt{Example Workflow:}

\begin{enumerate}
  \item Compute solutions globally:
        \verb|PA.center, PA.through, n = C.OC:CPP(z.A, z.B)|
  \item Draw all circles:
        \verb|\tkzDrawCirclesFromPaths(PA.center, PA.through)|
  \item Export the same centers and points to nodes:

        \verb|\directlua{ tkz.nodes_from_paths(PA.center, PA.through) }|
  \item Now highlight one circle, for instance:
        \verb|\draw[red,thick] (z.w1) -- (z.t1);|
\end{enumerate}

\medskip
In summary, \verb|PA.xxx| must be \emph{global} for TikZ macros,
while \verb|tkz.nodes_from_paths| lets you create node names (\verb|w1,t1,...|) from
either global or local paths depending on how much control you need.



\subsection{Case: \tkzMeth{point}{PPP}(p, p)}
\label{sub:PPP}

\texttt{Purpose: }

The method \tkzname{PPP} constructs the \emph{circumcircle} passing through
three distinct points: the current point (\tkzname{self}) and the two given
points \tkzname{a} and \tkzname{b}.

If the three points are collinear or not distinct, no circle can be defined and
the method returns empty paths with a counter equal to~0.

\medskip
\texttt{Syntax: } \verb|PA.center, PA.through, n = z.A:PPP(z.B, z.C)|

\medskip
\texttt{Returns: }

\begin{itemize}
  \item \tkzname{PA.center}: a \tkzname{path} containing the center(s) of the circumcircle(s);
  \item \tkzname{PA.through}: a \tkzname{path} containing the corresponding point(s) on each circle;
  \item \tkzname{n}: the number of valid circles (1 if a circumcircle exists, 0 otherwise).
\end{itemize}

\medskip
\texttt{Example usage: }

\begin{tkzexample}[latex=.45\textwidth]
\directlua{
  z.A = point(0, 0)
  z.B = point(4, 0)
  z.C = point(3, 4)
  PA = {}
  PA.center, PA.through, _ = z.A:PPP(z.B, z.C)
  tkz.nodes_from_paths(PA.center, PA.through)
  % w and t by default
}
\begin{center}
  \begin{tikzpicture}
    \tkzGetNodes
    \tkzDrawPolygon(A,B,C)
    \tkzDrawCircles(w1,t1)
    \tkzDrawPoints(A,B,C)
  \end{tikzpicture}
\end{center}
\end{tkzexample}

\medskip
\texttt{Remarks: }
This is the simplest case of the Apollonius family.
If \tkzname{n = 0}, the three points are collinear and no circle is drawn.

%
% ---

\subsection{Case: \tkzMeth{line}{LPP}(p, p)}
\label{sub:LPP}

\medskip
\texttt{Purpose: } Given a running line~$L$ (the current object \code{self}) and two points \code{a} and \code{b} lying on the \emph{same side} of~$L$, this method constructs the circle(s) tangent to~$L$ and passing through both \code{a} and \code{b}.

\medskip

\texttt{Syntax: }
\verb|local PA.center, PA.through, n = L:LPP(a, b)|

\medskip
\texttt{Parameters: }

\begin{itemize}
  \item \code{a}, \code{b} : two distinct points.
  \item \emph{Precondition:} \code{a} and \code{b} are on the same side of the running line \code{self}.%
\end{itemize}

\medskip
The method is designed to handle a variety of geometric configurations, including the following special cases:

\begin{itemize}
  \item The segment $[MN]$ is perpendicular to the line $(AB)$;
  \item The segment $[MN]$ is parallel to the line $(AB)$;
  \item The points $M$ and $N$ lie on opposite sides of the line $(AB)$;
  \item One of the points (either $M$ or $N$) lies on the line $(AB)$.
\end{itemize}

This construction is useful in problems involving constrained circle placements or Apollonius-type configurations.

\texttt{Returns: }

\begin{itemize}
  \item \texttt{PA.center} and \texttt{PA.through} : two paths containing respectively the centers and the points of tangency of the solution circles.
  \item \texttt{n} : the number of existing solutions (\verb|0|, \verb|1| or \verb|2|).
\end{itemize}

\medskip
\texttt{Remarks: }

\begin{itemize}
  \item If \code{n = 0}, both paths are empty and no circle can be drawn.
  \item In degenerate cases with a single solution, the same circle is recorded once (\code{n = 1}).
  \item No \verb|tex.error| is raised; the method always returns valid paths.
  \item The results can be transferred and drawn using the macros:

   \verb|\tkzNodesFromPaths| and \verb|\tkzDrawCirclesFromPaths|.
\end{itemize}

\medskip

\texttt{Existence \& number of solutions}
\begin{itemize}
  \item \texttt{No solution (0).} If the intersection point $i=L\cap (ab)$ exists and lies on the segment $[a,b]$, then no circle tangent to $L$ can pass through $a$ and $b$.
  \item \texttt{Single solution (1).}
    \begin{itemize}
      \item If exactly one of the points \code{a} or \code{b} belongs to $L$ (point of tangency), the unique solution has its center at the intersection of the mediator of $[a,b]$ with the perpendicular to $L$ at that point.
      \item If $L \parallel (ab)$, the unique solution has its center at the circumcenter of the triangle $(a,b,p)$ where $p$ is the orthogonal projection of $\mathrm{mid}(a,b)$ on $L$.
    \end{itemize}
  \item \texttt{Two solutions (2).}
    \begin{itemize}
      \item If $L \perp (ab)$ and $i=L\cap (ab)$ is defined (and not on $[a,b]$), two symmetric solutions exist.
      \item In the general oblique case (neither parallel nor orthogonal), two solutions exist.
    \end{itemize}
\end{itemize}

\medskip

\texttt{Example usage: }

\vspace{1em}

\begin{tkzexample}[latex=.4\textwidth]
\directlua{
  init_elements()
  z.A = point(0, 0)
  z.B = point(6, 0)
  z.M = point(1, 1)
  z.N = point(2, 5)
  L.AB = line(z.A, z.B)
  PA.center,
  PA.through, n = L.AB:LPP(z.M, z.N)
  tkz.nodes_from_paths(PA.center,
                       PA.through, "O", "T")
}

\begin{tikzpicture}[scale = .4]
  \tkzGetNodes
  \tkzDrawLines(A,B M,N)
  \tkzDrawCircles(O1,T1 O2,T2)
  \tkzDrawPoints(A,B,M,N)
  \tkzLabelPoints(A,B,M,N)
\end{tikzpicture}
\end{tkzexample}

\vspace{1em}
Let's look at the case where the line $(MN)$ is parallel to the initial line.

\vspace{1em}
\begin{tkzexample}[latex=.4\textwidth]
\directlua{
  init_elements()
  z.A = point(0, 0)
  z.B = point(6, 0)
  z.M = point(0, 3)
  z.N = point(5, 3)
  L.AB = line(z.A, z.B)
  PA.center, PA.through,
          n = L.AB:LPP(z.M, z.N)}
\begin{center}
\begin{tikzpicture}[scale = .75]
 \tkzGetNodes
 \tkzDrawCirclesFromPaths[draw,
          red](PA.center,PA.through)
 \tkzDrawSegments(A,B M,N)
 \tkzDrawPoints(A,B,M,N)
 \tkzLabelPoints(A,B,M,N)
\end{tikzpicture}
\end{center}
\end{tkzexample}

\vspace{1em}
Where the line is perpendicular to the initial line.

\vspace{1em}

\begin{tkzexample}[latex=.5\textwidth]
\directlua{
 init_elements()
 z.A = point(0, 0)
 z.B = point(6, 0)
 z.M = point(1,  1)
 z.N = point(1, 5)
 L.AB = line(z.A, z.B)
 PA.center,
 PA.through, n = L.AB:LPP(z.M, z.N)}
\begin{tikzpicture}[scale =.6]
  \tkzGetNodes
  \tkzDrawLines(A,B M,N)
  \tkzDrawCirclesFromPaths[draw,
     orange](PA.center,PA.through)
  \tkzDrawPoints(A,B,M,N)
  \tkzLabelPoints(A,B,M,N)
\end{tikzpicture}
\end{tkzexample}

\vspace{1em}
The last special case is when one of the points is on the initial line. In this case, there's only one solution.

\vspace{1em}

\begin{tkzexample}[latex=.4\textwidth]
\directlua{
  init_elements()
  z.A = point(0, 0)
  z.B = point(5, 0)
  z.M = point(1, 0)
  z.N = point(3, 5)
  L.AB = line(z.A, z.B)
  PA.center,
  PA.through, n =L.AB:LPP(z.M, z.N)
  PA.center, PA.through,
  n = L.AB:LPP(z.M, z.N)}

\begin{tikzpicture}[scale =.6]
  \tkzGetNodes
  \tkzDrawCirclesFromPaths[draw,
    orange](PA.center,PA.through)
  \tkzDrawLines(A,B)
  \tkzDrawPoints(A,B,M,N)
  \tkzLabelPoints(A,B,M,N)
\end{tikzpicture}
\end{tkzexample}


\subsection{Case: \tkzMeth{line}{LLP}(L, p)}
\label{sub:LLP}

\texttt{Purpose: }
Given the running line~$D$ (\code{self}), another line~\code{L}, and a point~\code{p}, this method constructs the circle(s) passing through \code{p} and tangent to \emph{both} lines~$D$ and~$L$.

\medskip
\texttt{Syntax: }
\verb|PA.center, PA.through, n = L.AB:LLP(L, p)|

\medskip
Let us consider two straight lines, $(AB)$ and $(CD)$, and a point~$P$ not lying on either line.
The problem is to determine whether there exists a circle passing through~$P$ and tangent to both lines.

\texttt{Parameters: }
\begin{itemize}
  \item \code{L}: a line distinct from the running line~\code{self} (they may be parallel or intersecting).
  \item \code{p}: a point in the plane.
\end{itemize}

\texttt{Returns: }
\begin{itemize}
  \item \code{PA.center}, \code{PA.through}: two paths containing the centers and tangent points of the solution circles.
  \item \code{n}: the number of solutions (0, 1, or 2).
  \item If no solution exists, \code{n = 0} and both paths are empty.
\end{itemize}

\texttt{Remarks: }
This method generalizes Apollonius' problem in the case of two lines and one point.
It handles both parallel and intersecting lines, as well as special configurations where the point lies on a bisector or one of the lines.

\vspace{1em}
 \texttt{General case}

\begin{tkzexample}[latex=.5\textwidth]
\directlua{
  init_elements()
  z.A = point(-1, 2)
  z.B = point(2, -3)
  z.C = point(-1, -3)
  z.D = point(2, 1)
  L.AB = line(z.A, z.B)
  L.CD = line(z.C, z.D)
  z.S = intersection(L.AB, L.CD)
  L.SI = tkz.bisector(z.S, z.A, z.D)
  z.P = L.SI:point(-1)
  local centers,
   throughs, n = L.AB:LLP(L.CD, z.P)
  tkz.nodes_from_paths(centers, throughs)
}
\begin{center}
\begin{tikzpicture}
  \tkzGetNodes
  \tkzDrawLines[red](A,B C,D)
  \tkzDrawCircles[blue](w1,t1 w2,t2)
  \tkzDrawPoints(A,B,C,D,P,w1,t1,w2,t2)
  \tkzLabelPoints(A,B,C,D,P)
\end{tikzpicture}
\end{center}
\end{tkzexample}


\vspace{1em}
 \texttt{Point inside the angle}

\begin{tkzexample}[latex=.5\textwidth]
\directlua{
  init_elements()
  z.A = point(0, 0)
  z.B = point(6, 0)
  L.AB = line(z.A, z.B)
  z.C = point(6, 4)
  L.AC = line(z.A, z.C)
  z.P = point(3, 1.5)
  local centers,
  throughs, n = L.AB:LLP(L.AC, z.P)
  tkz.nodes_from_paths(centers, throughs)
}
\begin{center}
\begin{tikzpicture}[scale=.75]
  \tkzGetNodes
  \tkzDrawLines[thick](A,B A,C)
  \tkzDrawCircles[red](w1,t1 w2,t2)
  \tkzDrawPoints(A,B,C,P,w1,t1,w2,t2)
  \tkzLabelPoints(A,B,C,P)
\end{tikzpicture}
\end{center}
\end{tkzexample}

\medskip
\texttt{Point on a bisector}

\begin{tkzexample}[latex=.5\textwidth]
\directlua{
  init_elements()
  z.A = point(0, 0)
  z.B = point(6, 0)
  L.AB  = line(z.A, z.B)
  z.C = point(6, 4)
  L.AC  = line(z.A, z.C)
  L.bi  = tkz.bisector(z.A, z.B, z.C)
  z.P = L.bi:point(0.4)
  local centers,
    throughs, n = L.AB:LLP(L.AC, z.P)
  tkz.nodes_from_paths(centers, throughs)
}
\begin{center}
\begin{tikzpicture}[scale=.75]
  \tkzGetNodes
  \tkzDrawLines(A,B A,C A,P)
  \tkzDrawCircles(w1,t1 w2,t2)
  \tkzDrawPoints(A,B,C,P)
  \tkzLabelPoints(A,B,C,P)
\end{tikzpicture}
\end{center}
\end{tkzexample}

\medskip
 \texttt{Point on one of the lines}

\begin{tkzexample}[latex=.5\textwidth]
\directlua{
  init_elements()
  z.A = point(0, 0)
  z.B = point(4, -3)
  L.AB  = line(z.A, z.B)
  z.C = point(6, 4)
  L.AC  = line(z.A, z.C)
  z.P = L.AC:point(.3)
  local centers,
    throughs, n = L.AB:LLP(L.AC, z.P)
  tkz.nodes_from_paths(centers, throughs)
}
\begin{center}
\begin{tikzpicture}[scale=.6]
  \tkzGetNodes
  \tkzDrawLines(A,B A,C)
  \tkzDrawCircles(w1,t1 w2,t2)
  \tkzDrawPoints(A,B,C,P)
  \tkzLabelPoints(A,B,C,P)
\end{tikzpicture}
\end{center}
\end{tkzexample}


\vspace{1em}
\texttt{Parallel lines}

\begin{tkzexample}[latex=.5\textwidth]
\directlua{
  init_elements()
  z.A = point(-2, 2)
  z.B = point(1, -1)
  z.C = point(-2, 6)
  z.D = point(1, 3)
  L.AB = line(z.A, z.B)
  L.CD = line(z.C, z.D)
  z.P = point(-1, 2.5)
  local centers,
    throughs, n = L.AB:LLP(L.CD, z.P)
  tkz.nodes_from_paths(centers, throughs)
}
\begin{center}
\begin{tikzpicture}
  \tkzGetNodes
  \tkzDrawLines[red](A,B C,D)
  \tkzDrawCircles[blue](w1,t1 w2,t2)
  \tkzDrawPoints(A,B,C,D,P,w1,t1,w2,t2)
  \tkzLabelPoints(A,B,C,D,P)
\end{tikzpicture}
\end{center}
\end{tkzexample}


\subsection{Case: \tkzMeth{line}{LLL}(L, L)}
\label{sub:LLL}


This method completes the family of contact configurations.
It handles the case of a circle tangent to three lines.

\medskip
\texttt{Syntax: }

\begin{verbatim}
C_in = L.AB:LLL(L.BC, L.CA, "in")
C_exA = L.AB:LLL(L.BC, L.CA, "pa")
C_in, C_ea, C_eb, C_ec = L.AB:LLL(L.BC, L.CA)
\end{verbatim}

\medskip
\texttt{Purpose: }

The method \tkzname{LLL} constructs the circles tangent to three lines.

\begin{itemize}
  \item If the three lines are \emph{non-parallel}, they define a triangle,
  and \tkzname{LLL} returns the \emph{incircle} and the three \emph{excircles},
  equivalent to \tkzname{triangle:c\_lll("all")}.
  \item If exactly two lines are parallel and the third one is transversal,
  two circles are tangent to the three lines; these two solutions are returned.
\end{itemize}

\medskip
\texttt{Returns: }

\begin{itemize}
  \item In the triangle case:
    \begin{itemize}
      \item \texttt{"all"} → four circles (incircle and three excircles),
      \item \texttt{"in"} → incircle only,
      \item \texttt{"pa"}, \texttt{"pb"}, \texttt{"pc"} → the corresponding excircle.
    \end{itemize}
  \item In the parallel case: two circles tangent to the three lines.
\end{itemize}

\medskip
\texttt{Errors: }

If at least two pairs of lines are parallel, the configuration is degenerate.

\medskip
\texttt{Example usage: }

\begin{tkzexample}[latex=.5\textwidth]
\directlua{
  init_elements()
  z.A = point(0, 0)
  z.B = point(6, 0)
  z.C = point(4.5, 6)
  T.ABC = triangle(z.A, z.B, z.C)
  L.AB = line(z.A,z.B)
  L.AC = line(z.A,z.C)
  L.med = L.AB : mediator ()
  z.M = L.AB.mid
  z.x, z.y = get_points(L.med)
  z.H = L.AB:projection(z.C)
  L.ortho = L.AB:orthogonal_from(z.C)
  PA.center,
  PA.through = L.AC:LLL(L.med, L.ortho)
  z.w = PA.center:get(1)
  z.t = PA.through:get(1)
  PA.center,
  PA.through = L.AB:LLL(L.AC, T.ABC.bc)
  z.o = PA.center:get(1)
  z.h = PA.through:get(1)}
\begin{center}
  \begin{tikzpicture}
\tkzGetNodes
\tkzDrawLines(A,B A,C B,C)
\tkzDrawCircles(w,t o,h)
\tkzDrawPoints(A,B,C)
\tkzLabelPoints(A,B,C)
\end{tikzpicture}
  \end{center}
\end{tkzexample}

\subsection{Case: \tkzMeth{circle}{CLL}(L, L)}
\label{sub:CLL}

\texttt{Alias: } \verb|c_cll|

\medskip
\texttt{Syntax:}

\verb|PA.center, PA.through, n = circle:CLL(L1, L2, choice, inside)|

\medskip
\texttt{Purpose:}

Constructs the circles that are tangent to:
\begin{itemize}
  \item the current circle (\verb|self|),
  \item and the two lines \verb|L1| and \verb|L2|.
\end{itemize}

The method returns two \tkzname{path} objects containing all the solutions found:
\begin{itemize}
  \item \verb|PA.center| – centers of the solution circles,
  \item \verb|PA.through| – corresponding through-points (point of tangency),
  \item \verb|n| – number of solutions.
\end{itemize}

\medskip
\texttt{Options:}

\begin{itemize}
  \item \verb|choice| – Optional. Integer from \verb|1| to \verb|4| selecting one of the angular sectors formed by the two lines at their intersection.
  By default, \verb|choice = "all"| which means all sectors are explored and all valid solutions are returned.
  \item \verb|inside| – Optional. If set to \verb|"inside"|, the circle is tangent internally to the reference circle (its radius is considered negative in the auxiliary construction). If omitted, tangency is external.
\end{itemize}

\medskip
\texttt{Explanation:}

\begin{itemize}
  \item The intersection of \verb|L1| and \verb|L2| defines four sectors numbered from \verb|1| to \verb|4| in the direct (counterclockwise) order starting from \verb|L1|.
  \item For each chosen sector, up to two circles can be tangent to the two lines and to the reference circle.
  \item With the default option \verb|choice = "all"|, the method tries all four sectors and appends every valid solution to the paths.
\end{itemize}

\medskip
\texttt{Remarks:}

\begin{itemize}
  \item If the lines are parallel, this method does not apply.
  \item The user can draw the resulting circles using
        \verb|\tkzDrawCirclesFromPaths(PA.center, PA.through)|.
  \item The value of \verb|n| gives the total number of circles stored in the paths.
\end{itemize}

\vspace{1em}
\texttt{Examples usage: }

\vspace{1em}
\begin{tkzexample}[vbox]
\directlua{
  init_elements()
  z.A = point(0, 0)
  z.B = point(6, -2)
  L.AB  = line(z.A, z.B)
  z.C = point(-3, -4)
  z.D = point(3, 1)
  L.CD  = line(z.C, z.D)
  z.O = z.D + point(-3,1)
  z.X = z.O + point(0,1)
  C.OX = circle(z.O, z.X)
  PA.center, PA.through = C.OX:CLL(L.AB, L.CD,"all")
  tkz.nodes_from_paths(PA.center, PA.through)}
\begin{center}
\begin{tikzpicture}[scale=.75]
 \tkzGetNodes
 \tkzInit[xmin=-5,xmax=10,ymin=-1,ymax=5]
 \tkzClip
 \tkzDrawLines[blue,add =1 and .25](A,B)
 \tkzDrawLines[blue,add =.25 and 1](C,D)
 \tkzDrawPoints[blue,size=3](O)
 \tkzDrawCirclesFromPaths[draw,red,thick](PA.center,PA.through)
 \tkzDrawCircles[thick,blue](O,X)
 \tkzDrawPoints[red,size=3](w1)
 \tkzLabelPoints[red](O)
\end{tikzpicture}
\end{center}
\end{tkzexample}

\medskip
\texttt{Some explanations: } The given circle lies within sectors 1 and 2.
In this case, we are looking for the solution circles located in these two sectors.
There is a method that determines how a circle is positioned with respect to two lines: \verb|circle:lines_position(L1, L2)|.
If you do not know how many solutions exist, the following example shows how to proceed.


\vspace{1em}
\texttt{A special case: } The intersection point of the two lines belongs to the given circle. In this example, the \verb|CLL_all| method determines all solutions. The points that define the solution circles are stored in objects of type \texttt{path}.

\vspace{1em}
\begin{tkzexample}[vbox]
\directlua{
  init_elements()
  z.A = point(0, 0)
  z.B = point(6, -2)
  L.AB  = line(z.A, z.B)
  z.C = point(-3, -4)
  z.D = point(3, 1)
  L.CD  = line(z.C, z.D)
  z.I = intersection(L.AB, L.CD)
  z.O = z.I + point(1,1)
  z.X = z.I
  C.OX = circle(z.O, z.X)
  PA.center, PA.through = C.OX:CLL_all(L.AB, L.CD)
  z.w1 = PA.center:get(1)
  z.t1 = PA.through:get(1)}

\begin{tikzpicture}[scale=1]
 \tkzGetNodes
 \tkzInit[xmin=-5,xmax=8,ymin=-5,ymax=5]
 \tkzClip
 \tkzDrawLines[cyan,add = .25 and 1](A,B C,D)
 \tkzDrawCircles[thick,cyan](O,X)
 \tkzDrawCirclesFromPaths[draw,orange,thick](PA.center,PA.through)
 \tkzDrawCircles[black,thick](w1,t1)
 \tkzLabelPoints[red](I)
\end{tikzpicture}
\end{tkzexample}


\vspace{1em}
\texttt{Another example}

\directlua{
  init_elements()
  z.A = point(0, 0)
  z.B = point(6, -2)
  L.AB  = line(z.A, z.B)
  z.C = point(-3, -4)
  z.D = point(3, 1)
  L.CD  = line(z.C, z.D)
  z.O = z.D + point(3,1)
  z.X = z.O + point(0,1)
  C.OX = circle(z.O, z.X)
  PA.center, PA.through = C.OX:CLL_all(L.AB, L.CD)
  z.w1 = PA.center:get(2)
  z.t1 = PA.through:get(2)}

\begin{tikzpicture}
 \tkzGetNodes
 \tkzInit[xmin=-.25,xmax=12,ymin=-8,ymax=8]
 \tkzClip
 \tkzDrawLines[blue,add = 1 and 4](A,B C,D)
 \tkzDrawCircles[thick,cyan](O,X)
 \tkzDrawCirclesFromPaths[draw,orange,thick](PA.center,PA.through)
 \tkzDrawCircles[thick,purple](w1,t1)
\end{tikzpicture}


\subsection{Case: \tkzMeth{circle}{CPP}(p, p)}
\label{sub:CPP}

\texttt{Circle tangent to a given circle and passing through two points.}

\medskip
\texttt{Purpose: }

Construct the circle(s) tangent to a given circle (the receiver \verb|self|) and passing through two given points~\verb|a| and~\verb|b|.

The name \verb|CPP| stands for: \emph{circle tangent to one \underline{c}ircle and passing through two \underline{p}oints}.

\medskip
\texttt{Remarks: }
This method returns one or two solution circles, depending on the configuration.
A valid solution exists only if both points are on the \emph{same side} of the given circle (both inside or both outside).

\medskip
\texttt{Parameters: }

\begin{itemize}
  \item \verb|a|, \verb|b| — two distinct points through which the solution circles must pass.
\end{itemize}

\medskip
\texttt{Syntax: }
\verb|local PA.center, PA.through, n = C.OC:CPP(a, b)|

\medskip
\texttt{Returns: }

\begin{itemize}
  \item \verb|PA.center| — a \tkzname{path} containing the centers of the solution circles;
  \item \verb|PA.through| — a \tkzname{path} containing the corresponding points on each circle;
  \item \verb|n| — the number of solutions (0, 1, or 2).
\end{itemize}

\medskip
\texttt{Usage: }
After calling this method, the Lua function \verb|tkz.nodes_from_paths(PA_center, PA_through)|
creates the corresponding nodes \verb|z.w1, z.w2, ...| and \verb|z.t1, z.t2, ...| for TeX drawing.
You can then draw all circles within a \verb|tikzpicture| environment using:

\verb|\tkzDrawCirclesFromPaths(PA.center, PA.through)|.

\medskip
\texttt{Existence condition: }
A solution exists only if both points \verb|a| and \verb|b| are on the same side of the given circle (either both inside or both outside its disk).
If one point lies inside and the other outside, no solution is returned (\verb|n = 0|).

\vspace{1em}
\begin{minipage}{.6\textwidth}
\directlua{
  init_elements()
  z.A = point(5,4)
  z.B = point(3,0)
  z.O = point(0,0)
  z.C = point(1,0)
  C.OC = circle(z.O, z.C)
  PA.center, PA.through,
  n = C.OC:CPP(z.A, z.B)
}
\begin{center}
\begin{tikzpicture}[scale=.6]
  \tkzGetNodes
  \tkzDrawCircle[blue](O,C)
  \tkzDrawPoints(A,B,C,O)
  \tkzDrawCirclesFromPaths[draw,
  red](PA.center,PA.through)
  \tkzLabelPoints(O,C,A,B)
\end{tikzpicture}
\end{center}
\end{minipage}
\begin{minipage}{.4\textwidth}
\begin{tkzexample}[code only]
\directlua{
  init_elements()
  z.A = point(5,4)
  z.B = point(3,0)
  z.O = point(0,0)
  z.C = point(1,0)
  C.OC = circle(z.O, z.C)
  PA.center,
  PA.through,
  n = C.OC:CPP(z.A, z.B)
}
\begin{tikzpicture}
  \tkzGetNodes
  \tkzDrawCircle[blue](O,C)
  \tkzDrawPoints(A,B,C,O)
  \tkzDrawCirclesFromPaths[draw,
      red](PA.center,PA.through)
\end{tikzpicture}
\end{tkzexample}
\end{minipage}

\vspace{1em}
\texttt{Special cases: }

\texttt{Equidistant points from the center: }

\begin{tkzexample}[latex=.5\textwidth]
\directlua{
  init_elements()
  z.A = point(2,3)
  z.B = point(2,-3)
  z.O = point(0,0)
  z.C = point(1,0)
  C.OC = circle(z.O, z.C)
  PA.center,
  PA.through,
  n = C.OC:CPP(z.A, z.B)}
\begin{center}
\begin{tikzpicture}[scale=.5]
  \tkzGetNodes
  \tkzDrawCircle[blue](O,C)
  \tkzDrawPoints(A,B,C,O)
  \tkzDrawCirclesFromPaths[draw,
  red](PA.center,PA.through)
  \tkzLabelPoints(O,C,A,B)
\end{tikzpicture}
\end{center}
\end{tkzexample}


\texttt{Tangent case: }
When the line $(AB)$ is tangent to the initial circle, there may be a single (degenerate) solution.
It can even happen that $(AB)$ is tangent and both points are equidistant from the center.

\begin{tkzexample}[latex=.5\textwidth]
\directlua{
  init_elements()
  z.A = point(1,3)
  z.B = point(1,-3)
  z.O = point(0,0)
  z.C = point(1,0)
  C.OC = circle(z.O, z.C)
  PA.center, PA.through,
  n = C.OC:CPP(z.A, z.B)}
\begin{center}
\begin{tikzpicture}[scale=.75]
  \tkzGetNodes
  \tkzDrawCircle[blue](O,C)
  \tkzDrawPoints(A,B,C,O)
  \tkzDrawCirclesFromPaths[draw,
  red](PA.center,PA.through)
  \tkzLabelPoints(O,C,A,B)
\end{tikzpicture}
\end{center}
\end{tkzexample}


\subsection{Case: \tkzMeth{circle}{CLP}(L, p, mode)}
\label{sub:CLP}

\texttt{Purpose: } Circle tangent to a given line and to the current circle, passing through a given point.

\medskip

\texttt{Syntax: } \verb|PA.center, PA.through, n = circle:CLP(L, p [, mode])|

\begin{itemize}
  \item \code{L} — a line object.
  \item \code{p} — a point through which the solution circle(s) must pass.
  \item \code{mode} — optional string in \code{"all"}, \code{"external"}, \code{"internal"}. Default is "all".

\end{itemize}

\texttt{Returns: } three values:
\begin{enumerate}
  \item \code{PA.center} — path of centers,
  \item \code{PA.through} — path of through-points,
  \item \code{n} — number of solutions ($0\dots4$).
\end{enumerate}

These paths can be drawn directly with:
\verb|\tkzDrawCirclesFromPaths(PA.center, PA.through)|.

%-------------------------------------------------
\texttt{Existence and number of solutions}

\begin{itemize}
  \item \emph{Generic case} (point not on line nor circle): up to \textbf{4} circles.
  \item \emph{Special cases} (only 2 or 0 solutions):
    \begin{itemize}
      \item $p$ on the circle → 2 solutions,
      \item $p$ on the line → 2 solutions,
      \item line tangent to the circle → one family may collapse,
      \item $p$ at line–circle intersection → no non-trivial solution.
    \end{itemize}
\end{itemize}

\texttt{Options}

\begin{itemize}
  \item \code{"all"} (default) — internal and external families (up to 4),
  \item \code{"external"} — only externally tangent solutions,
  \item \code{"internal"} — only internally tangent solutions (\code{"inside"} accepted).
\end{itemize}


\vspace{1em}
\texttt{Examples usage}

\begin{tkzexample}[latex=.5\textwidth]
\directlua{
  init_elements()
  z.A = point(0, 0)
  z.B = point(4, 0)
  L.AB = line (z.A, z.B)
  z.O = point(3, 3)
  z.T = point(3, 2)
  z.P = point(2, .25)
  C.OT = circle(z.O, z.T)
  PA.center, PA.through = C.OT:CLP(L.AB, z.P)
  z.O1 =  PA.center:get(1)
  z.O2 =  PA.center:get(2)
  PA.center,
  PA.through = C.OT:CLP(L.AB, z.P,'internal')
  z.O3 = PA.center:get(1)
  z.O4 = PA.center:get(2)}
\begin{tikzpicture}
  \tkzGetNodes
  \tkzDrawCircles[thick](O,T)
  \tkzDrawCircles[red](O1,P O2,P)
  \tkzDrawCircles[cyan](O3,P O4,P)
  \tkzDrawLines[thick](A,B)
  \tkzDrawPoints[size = 2](P)
  \tkzDrawPoints(A,B,O,O1,O2,O3,O4)
\end{tikzpicture}
\end{tkzexample}

\vspace{1em}
\begin{tkzexample}[latex=.5\textwidth]
\directlua{
 init_elements()
 z.A = point(0, 0)
 z.B = point(4, 0)
 L.AB = line (z.A, z.B)
 z.O = point(3, 2)
 z.T = point(3, 1)
 C.OT = circle(z.O, z.T)
 PA.center,
 PA.through = C.OT:CLP(L.AB, z.A, "internal")
 z.O1 = PA.center:get(1)
 z.O2 = PA.center:get(2)}
\begin{center}
\begin{tikzpicture}[scale=.5]
  \tkzGetNodes
  \tkzDrawCircles[thick](O,T)
  \tkzDrawCircles[red](O1,A O2,A)
  \tkzDrawLines[thick](A,B)
  \tkzDrawPoints[size = 3](A)
  \tkzDrawPoints(B,O,O1,O2)
\end{tikzpicture}
\end{center}
\end{tkzexample}

\vspace{1em}
\begin{tkzexample}[latex=.5\textwidth]
\directlua{
 init_elements()
 z.A = point(0, 0)
 z.B = point(4, 0)
 L.AB = line(z.A, z.B)
 z.O = point(3, 2)
 z.T = point(2, 2)
 C.OT = circle(z.O, z.T)
 z.P = C.OT:point(0.1)
 C.OT  = circle (z.O , z.T)
 PA.center,
 PA.through = C.OT:CLP(L.AB, z.P)}
\begin{center}
\begin{tikzpicture}[scale=.6]
  \tkzGetNodes
  \tkzDrawCircles[thick](O,T)
  \tkzDrawCirclesFromPaths[draw,
     red,thick](PA.center,PA.through)
  \tkzDrawLines[thick](A,B)
  \tkzDrawPoints[size = 3](P)
\end{tikzpicture}
\end{center}
\end{tkzexample}

%% ====================
%% END of CLP
%% ====================

\subsection{Case: \tkzMeth{circle}{CCP}(C, p, mode)}
\label{sub:CCP}

\texttt{Purpose: } Circle tangent to two given circles and passing through a point.

This method constructs 0, 2, or 4 circles depending on the configuration, passing through a given point $p$ and tangent to two given circles.


\medskip

The name \code{CCP} stands for:

\begin{center}
\texttt{c}ircle tangent to two \texttt{c}ircles and passing through a \texttt{p}oint.
\end{center}

medskip
\texttt{Syntax: } \verb|C1, C2 = C.A:CCP(C.B, p)        -- external solutions|

\verb|C3, C4 = C.A:CCP(C.B, p, "internal")  -- internal solutions|

\medskip

The general problem may admit up to four solutions: two circles tangent along the \emph{external} common tangents, and two circles tangent along the \emph{internal} common tangents.
For convenience, the method accepts an option:

\begin{itemize}
  \item \code{"external"} (default): returns the two solutions associated with the external tangents,
  \item \code{"internal"}: returns the two solutions associated with the internal tangents.
\end{itemize}


\medskip

\texttt{Special cases: }
\begin{itemize}
  \item If the point $p$ lies strictly inside or outside both disks, up to four solutions exist in total (two external, two internal).
  \item If $p$ lies on the circumference of one of the given circles, then $p$ is already a contact point; only two solutions remain.
  \item If $p$ lies simultaneously on both circles (intersection points), the configuration is degenerate and usually no non-trivial solution exists.
\end{itemize}



\medskip
\texttt{Examples usage:}


\begin{tkzexample}[latex=.5\textwidth]
\directlua{
  init_elements()
  z.A = point(0, 0)
  z.TA = point(3, 0)
  z.B = point(6, 2)
  z.TB = point(6, 1)
  z.P = point(3, 6)
  C.A = circle(z.A, z.TA)
  C.B = circle(z.B, z.TB)
  PA.center, PA.through  = C.A:CCP(C.B,
    z.P,"external")
  z.O1 = PA.center:get(1)
  z.O2 = PA.center:get(2)
  z.T1 = PA.through:get(1)
  z.T2 = PA.through:get(2)
  PA.center, PA.through = C.A:CCP(C.B,
    z.P,"internal")
  z.O3 = PA.center:get(1)
  z.O4 = PA.center:get(2)
  z.T3 = PA.through:get(1)
  z.T4 = PA.through:get(2)}
\begin{center}
  \begin{tikzpicture}[scale =.5]
 \tkzGetNodes
 \tkzDrawCircles[thick](A,TA B,TB)
 \tkzDrawCircles[red](O1,T1 O2,T2)
 \tkzDrawCircles[blue](O3,T3 O4,T4)
 \tkzDrawPoints[size=3](P)
 \tkzLabelPoints[above](P)
\end{tikzpicture}
\end{center}
\end{tkzexample}

\vspace{1em}
\texttt{The point is inside a circle}

\begin{tkzexample}[latex = .5\textwidth]
\directlua{
 init_elements()
 z.A = point(0, 0)
 z.TA = point(3, 0)
 z.B = point(4, 0)
 L.AB = line(z.A, z.B)
 z.TB = point(6, 0)
 C.A = circle(z.A, z.TA)
 C.B = circle(z.B, z.TB)
 z.P = point(3.45, 0.5)
 z.X, z.Y = intersection(C.A,C.B)
 PA.center, PA.through = C.A:CCP(C.B, z.P)
 tkz.nodes_from_paths(PA.center, PA.through)}
\begin{center}
\begin{tikzpicture}[scale=.5]
 \tkzGetNodes
 \tkzDrawCircles(A,TA B,TB)
 \tkzDrawCircles[red](w1,t1 w2,t2)
 \tkzDrawPoints(P)
\end{tikzpicture}
\end{center}
\end{tkzexample}

\vspace{1em}
\texttt{The point is on a circle}

\begin{tkzexample}[latex = .5\textwidth]
\directlua{
 init_elements()
 z.A = point(0, 0)
 z.a = point(3, 0)
 z.B = point(4, 0)
 z.b = z.B + point(0,2)
 L.AB = line(z.A, z.B)
 C.Aa = circle(z.A, z.a)
 C.Bb = circle(z.B, z.b)
 z.P = C.Bb:point(0.3)
 PA.center,
 PA.through,n = C.Aa:CCP(C.Bb, z.P)
}

\begin{center}
\begin{tikzpicture}[scale =.6]
 \tkzInit[xmin=-7,xmax=9,ymin=-9,ymax=8]
 \tkzClip \tkzGetNodes
 \tkzDrawCircles[thick,blue](A,a B,b)
 \tkzDrawPoints(A,B,P)
 \tkzDrawCirclesFromPaths[draw,
 purple,thick](PA.center, PA.through)
 \tkzLabelPoints(A,B,P)
\end{tikzpicture}
\end{center}
\end{tkzexample}

\subsection{Case: \tkzMeth{circle}{CCL}(C2, L)}
\label{sub:CCL}

\texttt{Purpose: } Find all circles that are tangent to:
\begin{itemize}
  \item the current circle \code{self} (called \code{C1}),
  \item another circle \code{C2},
  \item and a line \code{L}.
\end{itemize}

This method returns all possible solution circles (up to four).
\medskip
\texttt{Syntax:}
\begin{verbatim}
PA.center, PA.through, n = C1:CCL(C2, L)
\end{verbatim}

\begin{itemize}
  \item \code{C1} — the circle on which the method is applied (\code{self}),
  \item \code{C2} — another circle to which the solution circle must be tangent,
  \item \code{L} — a line object.
\end{itemize}

\medskip
\texttt{Returns:}
\begin{itemize}
  \item \code{pc} — a \code{path} containing the centers of the solution circles,
  \item \code{pt} — a \code{path} containing the corresponding tangency points on \code{C1},
  \item \code{n} — the number of solutions (from 0 to 4).
\end{itemize}


\medskip
\texttt{Special cases.}
\begin{itemize}
  \item If one of the circles is already tangent to \code{L} at a point \code{p}, any solution circle must also pass through \code{p}.
  \item If the line \code{L} intersects \code{C1}, no circle can be tangent to both \code{C1} and \code{L} on that side.
  \item If all three objects are mutually tangent, the configuration is degenerate and may admit a single or no solution.
\end{itemize}


\medskip
This method works with:
\begin{itemize}
  \item \verb|tkz.nodes_from_paths(pc, pt)| to name the solution points \code{(w1,t1)}, \code{(w2,t2)}, etc.,
  \item \code{\textbackslash tkzDrawCirclesFromPaths(pc, pt)} to draw all the solution circles.
\end{itemize}

\vspace{1em}
\texttt{Example A: }

\begin{tkzexample}[vbox]
\directlua{
  z.A = point(-4, 4)
  z.B = z.A + point(3, 0)
  C.AB  = circle(z.A, z.B)
  z.C = point(3,2)
  z.D = z.C + point(1,0)
  C.CD = circle(z.C, z.D)
  z.E = point(-3, 0)
  z.F = point(5, 0)
  L.EF  = line(z.E, z.F)
  PA.center, PA.through,n = C.AB:CCL(C.CD, L.EF)
  tkz.nodes_from_paths(PA.center, PA.through)
  }
\begin{center}
  \begin{tikzpicture}[scale=.5]
\tkzGetNodes
 \tkzInit[xmin=-8,xmax=8,ymin=-8,ymax=8]
 \tkzClip
 \tkzDrawLines[thick,blue](E,F)
 \tkzDrawCircles[thick,blue](A,B C,D)
 \tkzDrawCirclesFromPaths[draw,purple,thick](PA.center, PA.through)
\end{tikzpicture}
  \end{center}
\end{tkzexample}

\vspace{1em}
\texttt{Example B: }

\begin{tkzexample}[vbox]
\directlua{
 z.A = point(1, 5)
 z.B = point(6, 2)
 C.A = circle(from_radius(z.A, 4))
 L.AB = line(z.A, z.B)
 z.T = intersection(L.AB, C.A, {near = z.B})
 C.B = circle(z.B, z.T)
 L.TG1, L.TG2 = C.A:common_tangent(C.B,"external")
 z.E, z.F = L.TG2:get()
 PA.center, _, _ = C.A:CCL(C.B, L.TG2)
 z.C = PA.center:get(1)
 z.G = L.TG2:projection(z.C)}
\begin{center}
 \begin{tikzpicture}[scale = 1]
  \tkzInit[xmin=-3,xmax=8,ymin=-1,ymax=10]
 \tkzClip
 \tkzGetNodes
 \tkzFillCircle[red!20](A,E)
 \tkzFillCircle[green!20](C,G)
 \tkzFillCircle[yellow!20](B,F)
 \tkzDrawCircles[thick](A,E B,F C,G)
 \tkzDrawLines(E,F)
  \tkzDrawPoints[red](A,B,C,T,E,F,G)
 \tkzLabelPoints[red](A,B,C,T,E,F,G)
\end{tikzpicture}
 \end{center}
\end{tkzexample}

\vspace{1em}
\texttt{Example C: }

\begin{tkzexample}[vbox]
\directlua{
 z.A = point(1, 5)
 z.B = point(6, 2)
 C.A = circle(from_radius(z.A, 4))
 L.AB = line(z.A, z.B)
 z.T = intersection(L.AB, C.A, {near = z.B})
 C.B = circle(z.B, z.T)
 L.TG1, L.TG2 = C.A:common_tangent(C.B,"external")
 z.E, z.F = L.TG2:get()
 PA.center, _, _ = C.A:CCL(C.B, L.TG2)
 z.C = PA.center:get(2)
 z.G = L.TG2:projection(z.C)}
\begin{center}
\begin{tikzpicture}[scale =.5]
  \tkzInit[xmin=-5,xmax=25,ymin=-4,ymax=16]
 \tkzClip
 \tkzGetNodes
 \tkzFillCircle[red!20](A,E)
 \tkzFillCircle[green!20](C,G)
 \tkzFillCircle[purple!20](B,F)
 \tkzDrawCircles[thick](A,E B,F C,G)
 \tkzDrawLines(E,G)
  \tkzDrawPoints[red](A,B,C,T,E,F,G)
 \tkzLabelPoints[red](A,B,C,T,E,F,G)
\end{tikzpicture}
 \end{center}
\end{tkzexample}

\subsection{Case: \tkzMeth{circle}{CCC}(C2, C3 [, opts])}
\label{sub:CCC}

\texttt{Purpose: } Circles tangent to three given circles \(C_1=\) \verb|self|, \(C_2\), \(C_3\) (up to 8 solutions). Implementation follows Viète’s reduction.

\medskip
\texttt{Syntax: }\verb|pc, pt, n = C1:CCC(C2, C3 [, opts])|

\begin{itemize}
  \item \code{C1} — the current circle (\verb|self|).
  \item \code{C2}, \code{C3} — other circle objects.
  \item \code{opts} — optional table of tolerances:
    \begin{itemize}
      \item \verb|opts.abs| (default \verb|1e-4|) — absolute tolerance.
      \item \verb|opts.rel| (default \verb|1e-6|) — relative tolerance.
    \end{itemize}
\end{itemize}

\texttt{Returns:}
\begin{enumerate}
  \item \verb|pc| — path of solution centers,
  \item \verb|pt| — path of corresponding “through” points,
  \item \verb|n| — number of solutions found (0 to 8).
\end{enumerate}


\texttt{Result use in TikZ.}

\begin{itemize}
  \item With global paths: store \verb|PA.center, PA.through|, then draw all circles using:

        \verb|\tkzDrawCirclesFromPaths(PA.center, PA.through)|.
  \item With local paths: use \verb|\tkz.nodes_from_paths(pc, pt)| to create nodes \verb|w1,t1|, \verb|w2,t2|, ...
\end{itemize}

\subsubsection*{Example A — Using global paths}
\begin{verbatim}
\directlua{
  PA.center, PA.through, tkzNbCircles = C.A:CCC(C.B, C.C)
}
\tkzDrawCirclesFromPaths(PA.center, PA.through)
\end{verbatim}

\subsubsection*{Example B — Using local paths}
\begin{verbatim}
\directlua{
  local pc, pt, n = C.A:CCC(C.B, C.C)
  tkz.nodes_from_paths(pc, pt)
}
% circles are now (w1,t1), (w2,t2), ...
\end{verbatim}

\vspace{1em}
\texttt{Example C}
\begin{tkzexample}[vbox]
\directlua{
init_elements()
z.A = point(0, 0)
z.B = point(6, 0)
z.C = point(2, 5)
z.a = z.A + point(3, 0)
z.b = z.B + point(2, 0)
z.c = z.C + point(1, 0)
C.Aa = circle(z.A, z.a)
C.Bb = circle(z.B, z.b)
C.Cc = circle(z.C, z.c)
PA.center, PA.through, n = C.Aa:CCC(C.Bb, C.Cc)
}
\begin{center}
\begin{tikzpicture}[ scale = .5,gridded]
  \tkzInit[xmin=-8,xmax=12,ymin=-8,ymax=10]
  \tkzClip
  \tkzGetNodes
  \tkzDrawCircles[thick,blue](A,a B,b C,c)
  \tkzDrawCirclesFromPaths[draw,
    purple,thick](PA.center, PA.through)
\end{tikzpicture}
\end{center}
\end{tkzexample}

\vspace{1em}
\texttt{Example D}
\begin{tkzexample}[vbox]
\directlua{
init_elements()
z.A = point(0, 0)
z.B = point(6, 0)
z.C = point(4, 5)
z.a = z.A + point(4, 0)
z.b = z.B + point(2, 0)
z.c = z.C + point(1, 0)
C.Aa = circle(z.A, z.a)
C.Bb = circle(z.B, z.b)
C.Cc = circle(z.C, z.c)
pcenter, pthrough, n = C.Aa:CCC(C.Bb, C.Cc)
tkz.nodes_from_paths(pcenter, pthrough)
}
\begin{center}
\begin{tikzpicture}[ scale = .5,gridded]
  \tkzInit[xmin=-8,xmax=12,ymin=-8,ymax=10]
  \tkzClip
  \tkzGetNodes
  \tkzDrawCircles[thick,blue](A,a B,b C,c)
  \tkzDrawCirclesFromPaths[draw,
  purple,thick](pcenter, pthrough)
\end{tikzpicture}
\end{center}
\end{tkzexample}

\vspace{1em}
\texttt{Example E}
\begin{tkzexample}[vbox]
\directlua{
init_elements()
z.A = point(0, 0)
z.B = point(5, 0)
z.C = point(4, 5)
z.a = z.A + point(4, 0)
z.b = z.B + point(2, 0)
z.c = z.C + point(1, 0)
C.Aa = circle(z.A, z.a)
C.Bb = circle(z.B, z.b)
C.Cc = circle(z.C, z.c)
PA.center, PA.through, n = C.Aa:CCC(C.Bb, C.Cc)
}
\begin{center}
\begin{tikzpicture}[ scale = .5,gridded]
  \tkzInit[xmin=-8,xmax=12,ymin=-8,ymax=10]
  \tkzClip
  \tkzGetNodes
  \tkzDrawCircles[thick,blue](A,a B,b C,c)
  \tkzDrawCirclesFromPaths[draw,purple,thick](PA.center, PA.through)
\end{tikzpicture}
\end{center}
\end{tkzexample}

\vspace{1em}
\texttt{Example F}
\begin{tkzexample}[vbox]
\directlua{
init_elements()
z.A = point(0, 0)
z.B = point(5, 0)
z.C = point(1, 0)
z.a = z.A + point(8, 0)
z.b = z.B + point(2, 0)
z.c = z.C + point(1, 0)
C.Aa = circle(z.A, z.a)
C.Bb = circle(z.B, z.b)
C.Cc = circle(z.C, z.c)
PA.center, PA.through, n = C.Aa:CCC(C.Bb, C.Cc)
tex.print(n)
}
\begin{center}

\begin{tikzpicture}[ scale = .5,gridded]
  \tkzInit[xmin=-8,xmax=12,ymin=-8,ymax=10]
  \tkzClip
  \tkzGetNodes
  \tkzDrawCircles[thick,blue](A,a B,b C,c)
  \tkzDrawCirclesFromPaths[draw,purple,thick](PA.center, PA.through)
\end{tikzpicture}
\end{center}
\end{tkzexample}

\subsection{Case: \tkzMeth{circle}{CCC\_gergonne}(C\_2, C\_3)}
\label{sub:CCC_gergonne}

\texttt{Purpose: }
Construct all circles tangent to three given circles using the classical
Gergonne construction: centers of similitude, radical axis, poles, and
contact points. This method is only valid for circles that are disjoint in pairs.

\medskip
\texttt{Syntax: }
\begin{verbatim}
PA.center, PA.through, n = C1:CCC_gergonne(C2, C3 [, opts])
\end{verbatim}

\texttt{Arguments: }
\begin{itemize}
  \item \code{C1} — the calling circle (self).
  \item \code{C2}, \code{C3} — two other circles.
  \item \code{opts.abs} — optional absolute tolerance for tangency tests
        (default = \code{tkz.epsilon}).
\end{itemize}

\texttt{Returns: }
\begin{itemize}
  \item \code{PA.center} — a \texttt{path} storing the centers of all solution circles,
  \item \code{PA.through} — a \texttt{path} storing one contact point on each solution circle,
  \item \code{n} — the number of solutions found (may range from 0 to 8).
\end{itemize}
Each circle solution is represented by the pair:
\[
(\,PA.center[i],\; PA.through[i]\,), \quad i = 1,\dots,n.
\]
It can be drawn using
\verb|\tkzDrawCirclesFromPaths(PA.center, PA.through)|.

\medskip
\texttt{Geometric principle:}
The construction follows Gergonne:
\begin{enumerate}
  \item Compute the external and internal centers of similitude of the three circles:
        \(J_{12}, J_{13}, J_{23}\) (external),
        \(I_{12}, I_{13}, I_{23}\) (internal).
  \item These six points are aligned on four special lines:
        one line through the three external centers
        (``JJJ''), and three lines passing through one external
        and the two internal centers (``JII'').
  \item The radical center \(R\) of the three circles is computed
        (intersection of two radical axes).
  \item For each of the four lines, compute its pole with respect to each circle.
        From \(R\), draw lines to these poles; each such line intersects each circle
        (in general) in two points: these are candidate contact points.
  \item For each triple of contact points (one on each circle),
        compute the circumcenter. This is a potential solution.
        Keep it only if it is tangent to all three circles
        (checked by \code{circle:act}).
\end{enumerate}

\medskip
\texttt{Remarks: }
\begin{itemize}
  \item This method is purely geometric and works well when the circles
        are disjoint or only tangent. If the three circles intersect
        (secant configuration), some poles or intersections may be undefined,
        and no solution is returned.
  \item The method complements more robust algebraic approaches
        (Viète-based solvers).
  \item The returned paths can be directly used in a TikZ picture.
\end{itemize}

\texttt{Example: }

\begin{tkzexample}[vbox]
\directlua{
   z.A = point(0, 0)
   z.a = z.A + point (1, 0)
   z.B = point (4, 1)
   z.b = z.B + point (2, 0)
   z.C = point (2, 6)
   z.c = z.C + point (3, 0)
   C.Aa = circle(z.A, z.a)
   C.Bb = circle(z.B, z.b)
   C.Cc = circle(z.C,z.c)
   PA.center, PA.through = C.Aa:CCC_gergonne(C.Bb, C.Cc)
}
\begin{center}
\begin{tikzpicture}[scale = 0.8]
  \tkzInit[xmin=-8,xmax=12,ymin=-8,ymax=10]
  \tkzClip
  \tkzGetNodes
  \tkzDrawCircles[fill=blue!15](A,a B,b C,c)
  \tkzDrawCirclesFromPaths[draw,purple,thick](PA.center, PA.through)
\end{tikzpicture}
\end{center}
\end{tkzexample}

\vspace{1em}

\texttt{Explanations: }

\begin{tkzexample}[vbox]
\directlua{
   z.A = point(0, 0)
   z.a = z.A + point (1, 0)
   z.B = point (4, 1)
   z.b = z.B + point (2, 0)
   z.C = point (2, 6)
   z.c = z.C + point (3, 0)
   C.Aa = circle(z.A, z.a)
   C.Bb = circle(z.B, z.b)
   C.Cc = circle(z.C,z.c)
   L.rAB = C.Aa: radical_axis (C.Bb)
   L.rAC = C.Aa: radical_axis (C.Cc)
   L.rBC = C.Bb: radical_axis (C.Cc)
   z.w = C.Aa :radical_center(C.Bb, C.Cc)
   z.x1, z.y1 = L.rAB:get()
   z.x2, z.y2 = L.rAC:get()
   z.x3, z.y3 = L.rBC:get()
   z.t = C.Aa:radical_circle(C.Bb,C.Cc).through
   z.I = C.Aa:external_similitude(C.Bb)
   z.J = C.Aa:external_similitude(C.Cc)
   z.K = C.Bb:external_similitude(C.Cc)
   L.IK = line(z.I, z.K)
   z.P = C.Aa:pole(L.IK)
   z.Q = C.Bb:pole(L.IK)
   z.R = C.Cc:pole(L.IK)
   z.a1, z.a2 = intersection_lc_(z.w, z.P, z.A, z.a)
   z.b1, z.b2 = intersection_lc_(z.w, z.Q, z.B, z.b)
   z.c1, z.c2 = intersection_lc_(z.w, z.R, z.C, z.c)
   z.w1 = circum_center_(z.a1, z.b1, z.c2)
   z.w2 = circum_center_(z.a2, z.b2, z.c1)
}
\begin{center}
\begin{tikzpicture}[scale = 0.8]
   \tkzInit[xmin=-8,xmax=12,ymin=-8,ymax=10]
   \tkzClip
   \tkzGetNodes
   \tkzDrawCircles[blue](A,a B,b C,c)
   \tkzDrawCircles[purple](w,t w1,a1 w2,a2)
   \tkzDrawLines[red](x1,y1 x2,y2 x3,y3)
   \tkzDrawLines[cyan](w,P w,Q w,R)
   \tkzDrawLines[black](I,K)
   \tkzDrawPoints(A,B,C,w,I,J,K,P,Q,R,a1,a2,b1,b2,c1,c2)
   \tkzLabelPoints[below right](A,B,C,w,I,J,K,P,Q,R,a1,a2,b1,b2,c1,c2)
\end{tikzpicture}
\end{center}
\end{tkzexample}
\endinput