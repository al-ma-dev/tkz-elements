\newpage
\section{Class \tkzClass{rectangle}} % (fold)
The variable \code{R} holds a table used to store triangles. It is optional, and you are free to choose the variable name. However, using \code{R} is a recommended convention for clarity and consistency. If you use a custom variable (e.g., rectangles), you must initialize it manually. The \code{init\_elements()} function reinitializes the \code{R} table if used.

\subsection{Rectangle attributes} % (fold)
\label{sub:rectangle_attributes}


Points are created in the direct direction. A test is performed to check whether the points form a rectangle, otherwise compilation is blocked.

\begin{mybox}
Creation | R.ABCD = rectangle : new (z.A,z.B,z.C,z.D)|
\end{mybox}

\begin{center}
  \bgroup
  \catcode`_=12
  \small
  \captionof{table}{rectangle attributes.}\label{rectangle:att}
  \begin{tabular}{lll}
  \toprule
  \textbf{Attributes}       & \textbf{Application} & \\
  \midrule
  \tkzAttr{rectangle}{pa}     & |z.A = R.ABCD.pa| & \\
  \tkzAttr{rectangle}{pb}     & |z.B = R.ABCD.pb| & \\
  \tkzAttr{rectangle}{pc}     & |z.C = R.ABCD.pc| & \\
  \tkzAttr{rectangle}{pd}     & |z.D = R.ABCD.pd| & \\
  \tkzAttr{rectangle}{type}   &  |R.ABCD.type= 'rectangle'|  &\\
  \tkzAttr{rectangle}{center} & |z.I = R.ABCD.center| & center of the rectangle\\
  \tkzAttr{rectangle}{length} &  |R.ABCD.length| & the length \\
  \tkzAttr{rectangle}{width}  &  |R.ABCD.width| & the width \\
  \tkzAttr{rectangle}{diagonal}  &  |R.ABCD.diagonal| & diagonal length\\
  \tkzAttr{rectangle}{ab}     &  |R.ABCD.ab|   &  line passing through two vertices   \\
  \tkzAttr{rectangle}{ac}     &  |R.ABCD.ca|   &  idem. \\
  \tkzAttr{rectangle}{ad}     &  |R.ABCD.ad|   &  idem. \\
  \tkzAttr{rectangle}{bc}     &  |R.ABCD.bc|   &  idem. \\
  \tkzAttr{rectangle}{bd}     &  |R.ABCD.bd|   &  idem. \\
  \tkzAttr{rectangle}{cd}     &  |R.ABCD.cd|   &  idem. \\
  \bottomrule
  \end{tabular}
  \egroup
\end{center}


\subsubsection{Example} % (fold)
\label{ssub:example}
\begin{minipage}{.5\textwidth}
\begin{verbatim}
\directlua{
 init_elements()
 z.A = point(0, 0)
 z.B = point(4, 0)
 z.C = point(4, 4)
 z.D = point(0, 4)
 R.new = rectangle(z.A, z.B, z.C, z.D)
 z.I = R.new.center}

\begin{tikzpicture}
\tkzGetNodes
\tkzDrawPolygon(A,B,C,D)
\tkzDrawPoints(A,B,C,D)
\tkzLabelPoints(A,B)
\tkzLabelPoints[above](C,D)
\tkzDrawPoints[red](I)
\end{tikzpicture}
\end{verbatim}
\end{minipage}
\hspace{\fill}\begin{minipage}{.5\textwidth}
\directlua{
 init_elements()
 z.A = point(0, 0)
 z.B = point(4, 0)
 z.C = point(4, 4)
 z.D = point(0, 4)
 R.new = rectangle(z.A, z.B, z.C, z.D)
 z.I = R.new.center}

\begin{tikzpicture}[scale =1.5]
 \tkzGetNodes
 \tkzDrawPolygon(A,B,C,D)
 \tkzDrawSegment[dashed](A,C)
 \tkzDrawPoints(A,B,C,D)
 \tkzLabelPoints(A,B)
 \tkzLabelPoints[above](C,D)
 \tkzDrawPoints[red](I)
 \tkzLabelPoint[right = 10pt](I){$I$\\ |R.new.center|}
 \tkzLabelSegment[sloped,above](C,D){|R.new.length| =    
    \pmpn{\tkzUseLua{R.new.length}}}
 \tkzLabelSegment[sloped,above](A,C){|R.new.diagonal| = 
   \pmpn{\tkzUseLua{R.new.diagonal}}}
   % \tkzUseLua{R.new.length} and \tkzUseLua{R.new.diagonal} to get the values.
\end{tikzpicture}
\end{minipage}
% subsubsection example (end)
% subsection rectangle_attributes (end)

\newpage
\subsection{Rectangle methods} % (fold)
\label{sub:rectangle_methods}

\begin{center}
  \bgroup
  \catcode`_=12
  \small
  \captionof{table}{Rectangle methods.}\label{rectangle:met}
  \begin{tabular}{lll}
  \toprule
  \textbf{Functions} & \textbf{Reference}  &  \\
  \midrule  
  \tkzFct{rectangle}{angle (zi, za, angle)} &\\
  \tkzFct{rectangle}{gold (za, zb)} & \\
  \tkzFct{rectangle}{diagonal (za, zc)} &\\
  \tkzFct{rectangle}{side (za, zb, d)} & \\
  \midrule 
  \textbf{Methods} & \textbf{Reference}  &  \\
  \toprule 
  \tkzMeth{rectangle}{new(za ,zb, zc, zd)} &\\
  \tkzMeth{rectangle}{get\_lengths ()} & \\
  \bottomrule %
  \end{tabular}
  \egroup
\end{center}

\subsubsection{Method \tkzMeth{rectangle}{new(pt,pt,pt,pt)}} % (fold)
\label{ssub:function_igfct_rectangle_new_pt_pt_pt_pt}
This function creates a square using four points. No test is performed, and verification is left to the user.

\begin{tkzexample}[latex=.5\textwidth]
  \directlua{
  z.A = point(0, 0)
  z.B = point(4, 0)
  z.C = point(4, 3)
  z.D = point(-2, -3)
  L.AB = line(z.A,z.B)
  L.CD = line(z.C,z.D)
  z.I = intersection(L.AB, L.CD)
  C.I = circle(through(z.I, 3.5))
  z.G,z.E = intersection(L.AB, C.I)
  z.F,z.H = intersection(L.CD, C.I)
  R.I = rectangle(z.E, z.F, z.G, z.H)
  z.X = R.I.ab:projection(z.I)}
  \begin{tikzpicture}
  \tkzGetNodes
  \tkzDrawPolygon(E,F,G,H)
  \tkzDrawPoints(A,B,C,D,I,E,F,G,H,X)
  \tkzLabelPoints(A,B,C,D,I,E,F,G,H,X)
  \tkzDrawPoints(I)
  \end{tikzpicture}
\end{tkzexample}


% subsubsection function_igfct_rectangle_new_pt_pt_pt_pt (end)

\subsubsection{Method \tkzMeth{rectangle}{angle(pt,pt,an)}} % (fold)
\label{ssub:angle_method}

|R.ang = rectangle : angle (z.I,z.A)| ; |z.A | vertex ; ang angle between 2 vertices


\begin{minipage}{.5\textwidth}
\begin{verbatim}
\directlua{
 init_elements()
 z.A = point(0, 0)
 z.B = point(4, 0)
 z.I = point(4, 3)
 P.ABCD = rectangle:angle(z.I, z.A,
          math.pi / 6)
 z.B = P.ABCD.pb
 z.C = P.ABCD.pc
 z.D = P.ABCD.pd}
\begin{tikzpicture}[scale   = .5]
\tkzGetNodes
\tkzDrawPolygon(A,B,C,D)
\tkzDrawPoints(A,B,C)
\tkzLabelPoints(A,B,C,D)
\tkzDrawPoints[new](I)
\end{tikzpicture}
\end{verbatim}
\end{minipage}
\begin{minipage}{.5\textwidth}
\directlua{
 init_elements()
 z.A = point(0, 0)
 z.B = point(4, 0)
 z.I = point(4, 3)
 P.ABCD = rectangle:angle(z.I, z.A, math.pi / 6)
 z.B = P.ABCD.pb
 z.C = P.ABCD.pc
 z.D = P.ABCD.pd}
\begin{tikzpicture}[scale  = .5]
\tkzGetNodes
\tkzDrawPolygon(A,B,C,D)
\tkzDrawPoints(A,B,C)
\tkzLabelPoints(A,B)
\tkzLabelPoints[above](C,D)
\tkzDrawPoints[new](I)
\tkzLabelSegment[sloped,above](A,B){%
    |rectangle: angle (z.C,z.A,math.pi/6)|}
\end{tikzpicture}
\end{minipage}
% subsubsection angle_method (end)

\subsubsection{Method \tkzMeth{rectangle}{side(pt,pt,d)}} % (fold)
\label{ssub:side_method}
\begin{minipage}{.5\textwidth}
\begin{verbatim}
\directlua{
 init_elements()
 z.A = point(0, 0)
 z.B = point(4, 3)
 R.side = rectangle:side(z.A, z.B, 3)
 z.C = R.side.pc
 z.D = R.side.pd
 z.I = R.side.center}
\begin{tikzpicture}
\tkzGetNodes
\tkzDrawPolygon(A,B,C,D)
\tkzDrawPoints(A,B,C,D)
\tkzLabelPoints(A,B)
\tkzLabelPoints[above](C,D)
\tkzDrawPoints[red](I)
\end{tikzpicture}
\end{verbatim}
\end{minipage}
\begin{minipage}{.5\textwidth}
\directlua{
 init_elements()
 z.A = point(0, 0)
 z.B = point(4, 3)
 R.side = rectangle:side(z.A, z.B, 3)
 z.C = R.side.pc
 z.D = R.side.pd
 z.I = R.side.center}
\begin{tikzpicture}
\tkzGetNodes
\tkzDrawPolygon(A,B,C,D)
\tkzDrawPoints(A,B,C,D)
\tkzLabelPoints(A,B)
\tkzLabelPoints[above](C,D)
\tkzDrawPoints[red](I)
\tkzLabelSegment[sloped,above](A,B){%
   |rectangle : side (z.A,z.B,3)|}
\end{tikzpicture}
\end{minipage}
% subsubsection side_method (end)

\subsubsection{Method \tkzMeth{rectangle}{diagonal(pt,pt)}} % (fold)
\label{ssub:diagonal_method}
\begin{minipage}{.5\textwidth}
\begin{verbatim}
\directlua{
init_elements()
z.A = point(0, 0)
z.C = point(4, 3)
R.diag = rectangle:diagonal(z.A, z.C)
z.B = R.diag.pb
z.D = R.diag.pd
z.I = R.diag.center}
\begin{tikzpicture}
\tkzGetNodes
\tkzDrawPolygon(A,B,C,D)
\tkzDrawPoints(A,B,C,D)
\tkzLabelPoints(A,B)
\tkzLabelPoints[above](C,D)
\tkzDrawPoints[red](I)
\tkzLabelSegment[sloped,above](A,B){%
    |rectangle:diagonal(z.A,z.C)|}
\end{tikzpicture}
\end{verbatim}
\end{minipage}
\begin{minipage}{.5\textwidth}
\directlua{
 z.A = point(0, 0)
 z.C = point(4, 3)
 R.diag = rectangle:diagonal(z.A, z.C)
 z.B = R.diag.pb
 z.D = R.diag.pd
 z.I = R.diag.center}

\begin{tikzpicture}
\tkzGetNodes
\tkzDrawPolygon(A,B,C,D)
\tkzDrawPoints(A,B,C,D)
\tkzLabelPoints(A,B)
\tkzLabelPoints[above](C,D)
\tkzDrawPoints[red](I)
\tkzLabelSegment[sloped,above](A,B){%
   |rectangle : diagonal (z.A,z.C)|}
\end{tikzpicture}
\end{minipage}
% subsubsection diagonal_method (end)

\subsubsection{Method \tkzMeth{rectangle}{gold(pt,pt)}} % (fold)
\label{ssub:gold_method}
\begin{minipage}{.5\textwidth}
\begin{verbatim}
\directlua{
 init_elements()
 z.X = point(0, 0)
 z.Y = point(4, 2)
 R.gold = rectangle:gold(z.X, z.Y)
 z.Z = R.gold.pc
 z.W = R.gold.pd
 z.I = R.gold.center}
\begin{tikzpicture}
\tkzGetNodes
\tkzDrawPolygon(X,Y,Z,W)
\tkzDrawPoints(X,Y,Z,W)
\tkzLabelPoints(X,Y)
\tkzLabelPoints[above](Z,W)
\tkzDrawPoints[red](I)
\tkzLabelSegment[sloped,above](X,Y){%
|rectangle :  gold (z.X,z.Y)|}
\end{tikzpicture}
\end{verbatim}
\end{minipage}
\begin{minipage}{.5\textwidth}
\directlua{
init_elements()
z.X = point(0, 0)
z.Y = point(4, 2)
R.gold = rectangle:gold(z.X, z.Y)
z.Z = R.gold.pc
z.W = R.gold.pd
z.I = R.gold.center}

\begin{tikzpicture}
\tkzGetNodes
\tkzDrawPolygon(X,Y,Z,W)
\tkzDrawPoints(X,Y,Z,W)
\tkzLabelPoints(X,Y)
\tkzLabelPoints[above](Z,W)
\tkzDrawPoints[red](I)
\tkzLabelSegment[sloped,above](X,Y){%
  |rectangle :  gold (z.X,z.Y)|}
\end{tikzpicture}
\end{minipage}
% subsubsection gold_method (end)

\subsubsection{Method \tkzMeth{rectangle}{get\_lengths}()} % (fold)
\label{ssub:function_rectangle_get__lengths}

\begin{tkzexample}[latex=.45\textwidth]
\directlua{
 init_elements()
 z.I = point(2, 1)
 z.A = point(0, 0)
 R.ABCD = rectangle:angle(z.I, z.A, math.pi / 3)
 z.B = R.ABCD.pb
 z.C = R.ABCD.pc
 z.D = R.ABCD.pd
 tkzx,tkzy = R.ABCD:get_lengths()}
\begin{tikzpicture}
\tkzGetNodes
\tkzDrawPolygon(A,B,C,D)
\tkzDrawCircle(I,A)
\tkzDrawPoints(A,B,C,D)
\tkzLabelPoints(A,B)
\tkzLabelPoints[above](C,D)
\tkzDrawPoints[](I)
\tkzLabelSegment(A,B){%
    $\pmpn{\tkzUseLua{tkzx}}$}
\tkzLabelSegment[right](B,C){%
    $\pmpn{\tkzUseLua{tkzy}}$}
\end{tikzpicture}
\end{tkzexample}


% subsubsection function_rectangle_get__lengths (end)
% subsection rectangle_methods (end)