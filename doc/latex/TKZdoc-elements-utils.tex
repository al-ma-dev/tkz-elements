\newpage

\section{Module utils} % (fold)
\label{sec:utils}

The \tkzname{utils} module provides a collection of general-purpose utility functions used throughout the \tkzNamePack{tkz-elements} library. These functions are designed to support common tasks such as numerical rounding, type checking, floating-point comparisons, and table operations.

\vspace{1em}
Although these functions are not directly related to geometric constructions, they play a vital role in ensuring the consistency, robustness, and readability of the core algorithms. Most of them are small, efficient, and reusable in other contexts.

\vspace{1em}
This module is loaded automatically by \tkzNamePack{tkz-elements}, but its functions can also be used independently if needed.

\subsection{Table of module functions \tkzname{utils} } % (fold)
\label{sub:table_of_module_functions_tkzname_utils}

\begin{table}[htbp]
\centering
\caption{Functions of the module \tkzMod{utils}.}
\label{tab:utils:functions}
\begin{tabular}{@{}ll@{}}
\toprule
\textbf{Function} & \textbf{Reference} \\
\midrule

\tkzFct{utils}{utils.parse\_point(str)}        & [\ref{sub:function_utils_parse_point}] \\

\tkzFct{utils}{utils.format\_number(r, n)}     & [\ref{sub:function_utils_format_number}]\\

\tkzFct{utils}{utils.format\_coord(x, decimals)} & [\ref{sub:function_utils_format_coord}]\\

\tkzFct{utils}{utils.format\_point(z, decimals)} & [\ref{sub:function_utils_format_point}]\\

\tkzFct{utils}{utils.checknumber(x, decimals)}  & [\ref{sub:function_utils_checknumber}]\\

\tkzFct{utils}{utils.almost\_equal(a, b, eps)}  & [\ref{sub:function_utils_almost_equal}]\\

\tkzFct{utils}{utils.wlog(...)}              & [\ref{sub:function_utils_wlog}]\\
\bottomrule
\end{tabular}
\end{table}
% subsection table_of_module_functions_tkzname_utils (end)

\subsection{Function \tkzFct{utils}{parse\_point(str)}} % (fold)
\label{sub:function_utils_parse_point}

Parses a string of the form \code{"(x,y)"} and returns the corresponding numeric coordinates. This function supports optional spaces and scientific notation.

\paragraph{Syntax.}
\begin{center}
\code{local x, y = utils.parse\_point("(1.5, -2.3)")}
\end{center}

\paragraph{Description.}
The function takes a string argument and parses it to extract the \code{x} and \code{y} components as numbers. The input string must follow the format \code{"(x, y)"} where \code{x} and \code{y} can be floating-point values written in decimal or scientific notation.

\paragraph{Arguments.}
\begin{itemize}
\item \code{str} – A string representing a point, e.g., \code{"(3.5, -2.0)"}.
\end{itemize}

\paragraph{Returns.}
\code{x}, \code{y} – numeric coordinates as Lua numbers.
Two numerical values: the real and imaginary parts of the point.

\paragraph{Features.}
\begin{itemize}
\item Accepts optional spaces around numbers and commas.
\item Accepts scientific notation (\code{1e-2}, \code{3.4E+1}).
\item Raises an error for invalid formats.
\end{itemize}


\paragraph{Example.}
\begin{verbatim}
local x, y = utils.parse_point("(3.5, -2)")
-- x = 3.5, y = -2.0
\end{verbatim}

\paragraph{Related functions.}
\begin{itemize}
\item \tkzFct{utils}{format\_point(z, decimals)}
\item \tkzFct{utils}{format\_number(x, decimals)}
\end{itemize}

% subsection function_utils_parse_point (end)

\subsection{Function \tkzFct{utils}{format\_number(x, decimals)}} % (fold)
\label{sub:function_utils_format_number}

This function formats a numeric value (or a numeric string) into a string representation with a fixed number of decimal places.

\paragraph{Syntax.}
\begin{center}
\code{local str = utils.format\_number(math.pi, 3)}
\end{center}

\paragraph{Description.}
The function converts a number (or a string that can be converted to a number) into a string with the specified number of decimal digits. It is especially useful when generating clean numerical output for display or export to \TIKZ{} coordinates.

\paragraph{Arguments.}
\begin{itemize}
\item \code{x} – A number or a string convertible to a number.
\item \code{decimals} – Optional. The number of decimal places (default is 5).
\end{itemize}

\paragraph{Returns.}
A string representing the value of \code{x} with the specified number of decimals.

\paragraph{Features.}
\begin{itemize}
\item Automatically converts strings to numbers if possible.
\item Ensures consistent formatting for \TIKZ{} coordinates or LaTeX output.
\item Raises an error if the input is not valid.
\end{itemize}

\paragraph{Example.}
\begin{verbatim}
local a = utils.format_number(math.pi, 3)
% a = "3.142"

local b = utils.format_number("2.718281828", 2)
% b = "2.72"
\end{verbatim}

\paragraph{Error handling.}
An error is raised if \code{x} is not a valid number or numeric string.

\paragraph{Related functions.}
\begin{itemize}
\item \tkzFct{utils}{to\_decimal\_string(x, decimals)}
\item \tkzFct{utils}{format\_point(z, decimals)}
\end{itemize}

% subsection function_utils_format_number (end)


\subsection{Function \tkzFct{utils}{format\_coord(x, decimals)}} % (fold)
\label{sub:function_utils_format_coord}

This function formats a numerical value into a string with a fixed number of decimal places. It is a lighter version of \tkzFct{utils}{format\_number}, intended for internal use when inputs are guaranteed to be numeric.

\paragraph{Syntax.}
\begin{center}
\code{local s = utils.format\_coord(3.14159, 2)} \hfill $\rightarrow$ \code{"3.14"}
\end{center}

\paragraph{Arguments.}
\begin{itemize}
\item \code{x} – A number (not validated).
\item \code{decimals} – Optional number of decimal places (default: 5).
\end{itemize}

\paragraph{Returns.}
A string with fixed decimal formatting.

\paragraph{Notes.}
This function is used internally by \tkzFct{path}{add\_pair\_to\_path} and other path-building methods. Unlike \tkzFct{utils}{format\_number}, it does not perform input validation and should only be used with known numeric inputs.

\paragraph{Related functions.}
\begin{itemize}
\item \tkzFct{utils}{format\_number(x, decimals)} – safer alternative with validation
\end{itemize}

% subsection function_utils_format_coord (end)


\subsection{Function \tkzFct{utils}{checknumber(x, decimals)}} % (fold)
\label{sub:function_utils_checknumber}

Validates and converts a number or numeric string into a fixed-format decimal string.

\paragraph{Syntax.}
\begin{center}
\code{local s = utils.checknumber("2.71828", 4)} \hfill $\rightarrow$ \code{"2.7183"}
\end{center}

\paragraph{Arguments.}
\begin{itemize}
\item \code{x} – A number or numeric string.
\item \code{decimals} – Optional number of decimal digits (default: 5).
\end{itemize}

\paragraph{Returns.}
A formatted string representing the value rounded to the specified number of decimal places.

\paragraph{Remarks.}
Used internally to validate input before formatting. Returns an error if the input is not convertible.

\paragraph{Related functions.}
\begin{itemize}
\item \tkzFct{utils}{format\_number}
\end{itemize}

% subsection function_utils_checknumber (end)


\subsection{Function \tkzFct{utils}{format\_point(z, decimals)}} % (fold)
\label{sub:function_utils_format_point}

Converts a complex point into a string representation suitable for coordinate output.

\paragraph{Syntax.}
\begin{center}
\code{local s = utils.format\_point(z, 4)} \hfill $\rightarrow$ \code{"(1.0000,2.0000)"}
\end{center}

\paragraph{Arguments.}
\begin{itemize}
\item \code{z} – A table with fields \code{re} and \code{im}.
\item \code{decimals} – Optional precision (default: 5).
\end{itemize}

\paragraph{Returns.}
A string representing the point as \code{"(x,y)"}.

\paragraph{Error handling.}
Raises an error if \code{z} does not have numeric \code{re} and \code{im} components.

\paragraph{Related functions.}
\begin{itemize}
\item \tkzFct{utils}{format\_coord}
\end{itemize}

% subsection function_utils_format_point (end)

\subsection{Function \tkzFct{utils}{almost\_equal(a, b, epsilon)}} % (fold)
\label{sub:function_utils_almost_equal}

Returns \code{true} if two numbers are approximately equal within a given tolerance.

\paragraph{Syntax.}
\begin{center}
\code{if utils.almost\_equal(x, y) then ... end}
\end{center}

\paragraph{Arguments.}
\begin{itemize}
\item \code{a}, \code{b} – Two numbers to compare.
\item \code{epsilon} – Optional tolerance (default: \code{tkz\_epsilon}).
\end{itemize}

\paragraph{Returns.}
A boolean: \code{true} if the values differ by less than the tolerance.


\subsection{Function \tkzFct{utils}{wlog(...)}} % (fold)
\label{sub:function_utils_wlog}

Logs a formatted message to the .log file only, with a \code{[tkz-elements]} prefix.

\paragraph{Syntax.}
\begin{center}
\code{utils.wlog("Internal value: \%s", tostring(value))}
\end{center}

\paragraph{Returns.}
No return value. Logging only.


% section utils (end)
\endinput
