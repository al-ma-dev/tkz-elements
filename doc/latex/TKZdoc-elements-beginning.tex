\section{Getting started}

A quick introduction to get you started. We assume that the following packages are installed:  \tkzNamePack{tkz-euclide}, and \tkzNamePack{tkz-elements}.
Compile the following code using the \tkzEngine{lualatex} engine; you should obtain a line passing through points $A$ and $B$.

\begin{minipage}{.6\textwidth}
  \begin{tkzexample}[code only]
  % !TEX TS-program = lualatex
  \documentclass{article}
  \usepackage[mini]{tkz-euclide}
  \usepackage{tkz-elements}
  \begin{document}
  \directlua{
    init_elements()
    z.A = point(0, 1)
    % or z.A = point:new(0, 1)
    z.B = point(2, 0)
  }
  \begin{tikzpicture}
   \tkzGetNodes
   \tkzDrawLine(A,B)
   \tkzDrawPoints(A,B)
   \tkzLabelPoints(A,B)
  \end{tikzpicture}
  \end{document}
  \end{tkzexample}
\end{minipage}
\begin{minipage}{.4\textwidth}
  \directlua{
    z.A = point(0, 1)
    z.B = point(2, 0)
  }
  \begin{tikzpicture}
   \tkzGetNodes
   \tkzDrawLine(A,B)
   \tkzDrawPoints(A,B)
   \tkzLabelPoints(A,B)
  \end{tikzpicture}
\end{minipage}

To test your installation and follow the examples in this documentation, you need to load two packages: \tkzNamePack{tkz-euclide} and \tkzNamePack{tkz-elements}. The first package automatically loads \tkzNamePack{\TIKZ}, which is necessary for all graphical rendering.


The \tkzname{Lua} code is provided as an argument to the \tkzMacro{lualatex}{directlua} macro. I will often see  this code block as the \tkzname{Lua part}\footnote{This code can also be placed in an external file, e.g., \texttt{file.lua}.}. This part depends entirely on the \tkzNamePack{tkz-elements} package.

A crucial component in the \tkzEnv{tikz}{tikzpicture} environment is the macro \tkzMacro{tkz-elements}{tkzGetNodes}. This macro transfers the points defined in \tkzname{Lua} to \tkzNamePack{\TIKZ} by creating the corresponding nodes. All such points are stored in a table named \tkzname{z} and are accessed using the syntax \tkzname{z.label}. These labels are then reused within \tkzNamePack{tkz-euclide}.

When you define a point by assigning it a label and coordinates, it is internally represented as a complex number — the affix of the point. This representation allows the point to be located within an orthonormal Cartesian coordinate system.

If you want to use a different method for rendering your objects, this is the macro to modify. For example, Section~\ref{sec:metapost} presents \tkzMacro{tkz-elements}{tkzGetNodesMP}, a variant that enables communication with \code{MetaPost}.

Another essential element is the use of the function \tkzFct{tkz-elements}{init\_elements()}, which clears internal tables\footnote{All geometric objects are stored in Lua tables. These tables must be cleaned regularly, especially when creating multiple figures in sequence.} when working with multiple figures.

If everything worked correctly with the previous code, you're ready to begin creating geometric objects. Section~\ref{sec:class_and_object} introduces the available options and object structures.

Finally, it is important to be familiar with basic drawing commands in \tkzNamePack{tkz-euclide}, as they will be used to render the objects defined in \tkzname{Lua}.
\endinput